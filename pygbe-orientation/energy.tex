%!TEX root = CooperBarba2014.tex

We can decompose the total free energy into Coulombic, surface, and solvation energy:

\begin{equation}
F_\text{Total} = F_\text{Coulomb} + F_\text{surf} + F_\text{solv}.
\end{equation}

\medskip

\paragraph*{Coulombic energy---}

The Coulombic energy arises simply from the Coulomb interactions of all point charges. We compute it by

\begin{equation} \label{eq:coul_energy}
F_\text{Coulomb} = \frac{1}{2} \sum_j^{N_q}\sum^{N_q}_{\substack{i\\ i\neq j}} q_iq_j\frac{1}{4\pi |\mathbf{r}_i-\mathbf{r}_j|}
\end{equation}

\paragraph*{Solvation free energy---}

The solvation energy is the energy contribution of the protein's surroundings: solvent polarization, charged surfaces, and other proteins. We compute it as

\begin{align} \label{eq:solv_energy}
F_{\text{solv}} &= \frac{1}{2} \int_{\Omega} \rho \,(\phi_{\text{total}} - \phi_{\text{Coulomb}}) \\
&= \sum_{k=0}^{N_q} q_k (\phi_{\text{total}} - \phi_{\text{Coulomb}})(\mathbf{r}_k),
\end{align}

\noindent where $\rho$ is the charge distribution, consisting of point charges (which transforms the integral into a sum), and $\phi_\text{reac} = \phi_{\text{total}} - \phi_{\text{Coulomb}}$ is
%
\begin{equation} \label{eq:phi_reac_bem}
\phi_{\text{reac},\mathbf{r}_k} = -K_{L}^{\mathbf{r}_k}(\phi_{1,\Gamma_1}) + V_{L}^{\mathbf{r}_k} \left(\frac{\partial}{\partial \mathbf{n}}\phi_{1,\Gamma_1} \right) 
\end{equation}

\paragraph*{Surface free energy---}

We use the derivation of free energy of a surface with prescribed charge (like $\Gamma_2$ in Figure \ref{fig:molecule_surface}) from Chan and co-workers \cite{ChanMitchell1983,CarnieChan1993}. They describe the free energy on a surface as

\begin{equation} \label{eq:energy_surf}
F_\text{surf} = \frac{1}{2} \int_{\Gamma} G_c \sigma_0^2 d\Gamma, 
\end{equation} 

\noindent where $\phi = G_c \sigma_0$.
