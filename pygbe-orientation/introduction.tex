%!TEX root = CooperBarba-orientation.tex

Protein adsorption plays a key role in many biotechnological applications, particularly biomaterials and tissue engineering, biomedical implants and biosensors.
Yet, despite their importance, the specific mechanisms governing protein-surface interactions are not fully understood.\cite{Gray2004,RabeVerdesSeegel2011}

In the field of biosensors, protein adsorption needs to be engineered to obtain a successful device. 
Biosensors detect specific molecules using a nanoscale sensing element, such as a metallic nanoparticle or nanowire covered with a bioactive coating. 
The prevailing method to modify a sensor surface is via a self-assembled monolayer (\sam) of a small charged group, with ligand molecules layered on top to achieve the desired function. 
Antibodies are a common choice for the ligand molecules, although the newest devices use single-domain or single-chain fragment molecules.\cite{ByunETal2013,TrillingETal2014} 
Sensing occurs when a target biomolecule binds to the ligand molecule,  changing some physical parameter on the sensor, such as current in nanowires or plasmon resonance frequency in metallic nanoparticles. 

One of the factors crucially affecting biosensor performance is the orientation of ligand molecules.\cite{TajimaTakaiIshihara2011,TrillingBeekwilderZuilhof2013} 
These have specific binding sites, which need to be accessible to the target molecule for the biosensor to function well.
Probing protein orientation is thus one key goal of adsorption studies.
The aim of this study is to ascertain how orientation can be influenced by fabrication conditions regarding surface preparation, such as surface charge and ambient salt content. We consider in particular the antibody immunoglobulin G near a solid surface at different charge concentrations and ionic strengths. Using a smaller molecule (protein \gb), we could first confirm agreement of our results with published works reporting experiments\cite{BaioWeidnerBaughGambleStaytonCastner2012} and   simulations with a combined Monte Carlo and molecular dynamics method.\cite{LiuLiaoZhou2013} These previous works, among others, also concluded that electrostatic interactions are the dominant effect in the orientation of adsorbed proteins. In the case of immunoglobulin G (\ig), the protein is relevant for biosensor applications, but its large size would make all-atom molecular simulations quite cumbersome and expensive. For this reason, other researchers have studied adsorption of \ig using a coarse-grained model that considers each residue as a sphere (united-residue model),\cite{ZhouChenJiang2003} finding that electrostatics dominates the orientation for higher surface charges and that a positive charge can result in the desired ``tail-on'' placement for the \ig 1 iso-type, at low enough salt concentration. 
Here, we investigate the preferred orientations for the \ig 2a variant, which other other researchers found hard to control.
In addition to obtaining the preferred orientation at different conditions of charge and ionic strength, we also take a detailed look at the probability distribution in the parameter space.

Our model for protein-surface interactions uses the Poisson-Boltzmann equations in their integral formulation, representing the protein geometry as a dielectric interface in an implicit solvent. We recently verified the model against an analytical solution valid for spherical geometries and studied its numerical convergence in detail.\cite{CooperBarba2015a}
Previous studies on protein-surface interaction using the Poisson-Boltzmann equation showed that such a model is adequate as long as conformational changes in the protein are slight,\cite{YaoLenhoff2004,YaoLenhoff2005} and also that van der Waals effects can be neglected for realistic molecular geometries.\cite{RothNealLenhoff1996}
Conformational changes of the biomolecule can be ignored in this case because binding sites need to remain nearly unmodified during the biosensor fabrication process.\cite{TajimaTakaiIshihara2011} 
A continuum framework has been used in the past to study protein orientation,\cite{JufferArgosDevlieg1996} but it included ions explicitly. Other studies have used a coarse-grained model of the molecule, represented as  a set of spheres,\cite{ShengTsaoZhouJiang2002,ZhouTsaoShengJiang2004} assigned effective charges at the residue level,\cite{FreedCramer2011,ZhouChenJiang2003} or made approximations to account for pH effects.\cite{BiesheuvelvanderVeenNord2005,HartvigdeWeertOstergaartJorgensenJensen2011}

The sensor element (functionalized with the \sam) is represented in our model as a charged surface that interacts electrostatically with the biomolecule. A parameter sweep of the protein's rotation and tilt angles with respect to the solid surface provides energy landscapes, where the probability of finding the system in a given micro-state depends on the total free energy.
The continuum approach can thus provide insights to the conditions (surface charge and salt concentration) conducive to a favorable orientation of large proteins, too large for all-atom molecular simulation with today's computing power. It can also represent solid surfaces of any geometry, and we expect that it may in future assist in the design of better ligand-molecule immobilization techniques for high-sensitivity biosensors.

