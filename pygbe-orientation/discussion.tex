%!TEX root = CooperBarba-orientation.tex

\subsection{First case: protein G\,B1\,D4$^\prime$} \label{sec:disc_1PGB}

The orientation of protein \gb near charged surfaces was studied using a combined Monte Carlo and molecular dynamics method by Liu and co-workers\cite{LiuLiaoZhou2013} and experimentally by Baio and co-workers.\cite{BaioWeidnerBaughGambleStaytonCastner2012} The availability of these published results was a motivation to use this protein for a first test, to compare with the results obtained with our model. 

The results presented in Figure \ref{fig:1PGB_probability} show that for the most likely orientations, the dipole-moment vector is aligned with the vector normal to the interacting surface. This indicates that the dipole moment is the dominant effect that determines the protein's orientation, over local protein-surface  interactions. This is the expected result, since protein \gb is a relatively small biomolecule. 

Moreover, Figure \ref{fig:1PGB_probability} reveals that protein \gb behaves like a point dipole, as the most likely orientations shift 180$^\circ$ when the sign of the surface charge is flipped. This is also explained by the dipole moment dominating the orientation.

The dipolar behavior described by our results agrees with the experiments done by Baio and co-workers, \cite{BaioWeidnerBaughGambleStaytonCastner2012} in which they observed opposite orientations of protein \gb adsorbed on NH$_3^+$ and COO$^-$ self-assembled monolayers. With positively charged surfaces, most of the proteins oriented with the N-terminal of the protein pointing away from the surface, while for negatively charged surfaces the opposite occurred, with the C-terminal pointing away from the surface. This agrees with our results in Figure \ref{fig:1PGB_probability} (the dipole moment vector of protein \gb points from the C-terminal to the N-terminal).

Liu and co-workers \cite{LiuLiaoZhou2013} used a combined Monte Carlo and molecular dynamics method to obtain $<\cos(\alpha_{\text{tilt}})>=0.95$ for $\sigma = 0.05$C/m$^2$, and $<\cos(\alpha_{\text{tilt}})>=-0.85\pm0.05$ for $\sigma = -0.05$C/m$^2$, which agrees well with our results in Table \ref{table:avg}. Note that MD simulations consider van der Waals interactions and conformational changes of the protein, whereas these are not considered in our approach, explaining the slight differences in $<\cos(\alpha_{\text{tilt}})>$.
However, as noted by other researchers,\cite{ZhouChenJiang2003,BaioWeidnerBaughGambleStaytonCastner2012,LiuLiaoZhou2013} electrostatic effects often dominate protein-surface interactions and drive orientation during adsorption, while van der Waals effects play a role only in cases of very low surface charge. For example, in Ref.~\onlinecite{ZhouChenJiang2003}, van der Waals effects were of consequence in a setup with surface charge of 0.006C/m$^{2}$ and high ionic strength, leading to weak electrostatics. In a biosensor-fabrication scenario, this would only be the case with low-quality \sam s.

The results with protein \gb mean that an electrostatic solver with implicit solvent using the Poisson-Boltzmann equation is capable of capturing the driving mechanism of physical adsorption and orientation of the adsorbed molecule, at least in cases where the molecule's dipole moment is dominating the orientation. This is important because protein adsorption, being a free energy-driven process, is difficult to study experimentally\cite{MijajlovicETal2013} and thus simulations offer a promising alternative. Full atomistic molecular dynamics, however, demands large amounts of computing effort, and the possibility of using an electrostatics solver may extend the range of systems that can be investigated.

 \subsection{Second case: immunoglobulin G}
 
 With our numerical model already verified using an analytical solution for spherical geometry\cite{CooperBarba2015a} and the successful results for protein orientation of a small protein near a charged surface (Section \ref{sec:disc_1PGB}), we proceeded to study the effect of surface charge and salt concentration on the orientation of the antibody immunoglobulin G. Antibodies are widely used in biosensors as ligand molecules, due to their affinity and specificity with the target molecule (antigen), and it is vitally important that they are adsorbed on the sensor with the antigen-binding Ig fragment (Fab) pointing away from the sensor, into the oncoming flow containing the antigens (known as ``end-on'' or ``tail-on'' orientation).
Early experimental studies found that antigen/antibody ratio was especially low on negatively charged surfaces,\cite{BuijsETal1997} leading to the notion that protein orientation was affected to leading order by charge. 
One subsequent study\cite{ChenLiuZhouJiang2003} investigated the orientations of two iso-types of immunoglobulin G---IgG1, corresponding to \pdb\ structure {\small 1IGT}, and IgG2a, corresponding to \pdb\ {\small 1IGY}---adsorbed on positive and negatively charged surfaces. 
As an indirect method of probing antibody orientation, the researchers obtained adsorbed amounts and antigen/antibody ratios by means of surface-plasmon resonance experiments (e.g., a higher antigen/antibody ratio would indicate that more active sites are accessible and more antibodies are in a favorable orientation). 
The finding was that IgG1 mainly had a ``head-on'' (unfavorable) orientation on the negatively charged surfaces and a mix of ``tail-on'' (most favorable) and ``side-on'' orientations on the positively charged surfaces. 
IgG2a, on the other hand, had many orientations on both surfaces with positive and negative charge, leading to the conclusion that IgG2a is harder to control using electrostatic effects.
Results consistent with these were obtained by Zhou and co-workers\cite{ZhouChenJiang2003} using a united-residue model: a coarse-grained model where each amino-acid is treated as a sphere. They find that IgG1 will have the favorable ``end-on'' orientation on positive surfaces, as long as the charge density was large enough (0.018C/m/4$^{2}$, in their case) and the ionic strength was low. But IgG2a  did not show a clear preferred orientation at the conditions they looked at; the authors attribute this to the weaker dipole moment of this iso-type.
 
We investigated the orientation of IgG2a, which other studies found harder to orient favorably on a biosensor surface, and used two values of the surface charge ($\sigma=0.05$ and $0.2$C/m$^{2}$) and two values of salt concentration ($\kappa=0.125$ and $0.0315$\AA$^{-1}$), in each case varying four-fold.
 Figures \ref{fig:1IGT_negcharge} and \ref{fig:1IGT_poscharge} present the probability distribution of IgG2a for many orientations (given by $\alpha_\text{tilt}$ and $\alpha_\text{rot}$), in each case.
 The following discussion refers to each variation of the parameters and the effect on the preferred orientation of the adsorbed antibody and its probability.

 \medskip
 
 \paragraph*{Effect of surface charge---}
 
The lower value of surface charge here is $\sigma=\pm 0.05$C/m$^2$, the same value used in Ref.~\onlinecite{LiuLiaoZhou2013} to mimic the experiments reported in Ref.~\onlinecite{BaioWeidnerBaughGambleStaytonCastner2012}. Figures \ref{fig:1IGT_2D_sig-005} and \ref{fig:1IGT_2D_sig005} show that for the lower value of surface charge with the higher salt concentration ($\kappa=0.125$\AA$^{-1}$), there is no clear preferred orientation, to the point that the most likely orientation has a probability of barely around 10\%. This means that adsorbing the antibodies under these conditions would result in a wide range of orientations, which would not be favorable for biosensor fabrication. This observation is consistent with a previous study using a unified-residue model,\cite{ZhouChenJiang2003} where this particular antibody showed many possible orientations. The authors of that study attributed this behavior to the weaker dipole moment of this molecule, compared with the variant IgG1.
 
 With the higher value of surface charge, in this case $\sigma=\pm0.2$C/m$^2$, the orientation probability distribution in the case of negative charge shows no appreciable pattern of improvement.
 For positive surface charge, however, this higher value of $\sigma$ gives a markedly higher probability for the preferred orientation.
In the cases with higher salt concentration, Figure \ref{fig:1IGT_2D_sig020_kappa01250} shows a preferred orientation with a probability more than 5$\times$ higher than \ref{fig:1IGT_2D_sig005}, at the same tilt angle. For the lower value of salt concentration (Figure \ref{fig:1IGT_2D_sig020_kappa003125}), this effect is even larger, and it additionally shifts the preferred tilt angle from $44^{\circ}$ to $76^{\circ}$. Note that the dipole-moment vector does not point straight through the middle between the two Fab fragments, but in an angle.
 
The results presented on Figures \ref{fig:1IGT_negcharge} and \ref{fig:1IGT_poscharge} also show that increasing the surface charge shifts the tilt of the dipole-moment vector for the lower salt concentration with negative charge, and for the higher salt concentration with positive charge. This indicates that, in contrast to protein \gb, local interactions dominate over the dipole moment. If the dipole moment were the dominant effect, the dipole moment vector would tend to align to the surface normal as the surface charge increases.
This argues against the suggestion by other researchers\cite{ChenLiuZhouJiang2003,ZhouChenJiang2003} that the dipole-moment vector is the main determinant of orientation.
 
 \medskip
 
 \paragraph*{Effect of salt concentration---}
 
As the surface charge density was varied four-fold, we also varied the Debye length ($\kappa^{-1}$) four-fold. In terms of salt concentration, it means a 16$\times$ decrease in the amount of salt. The higher value of salt concentration corresponds to 145mM, which is in the physiological salt range.  
 
 Like increasing the surface charge, lowering the salt concentration affects the orientation probability distribution. For $\sigma=-0.05$C/m$^2$ (Fig.~\ref{fig:1IGT_2D_sig-005_kappa003125}), the effect is an increase in probability of the preferred orientation by nearly 2$\times$. For the positive weaker charge, $\sigma=\pm0.05$C/m$^2$, (Fig.~\ref{fig:1IGT_2D_sig005_kappa003125}), not only does the peak probability increase considerably, but the preferred tilt shifts from $64^{\circ}$ to $44^{\circ}$.
 For the stronger negative charge, $\sigma=-0.2$C/m$^2$, the probability peak increases to almost 100\% for an orientation where the dipole-moment vector is pointing down towards the surface in an unfavorable position (Fig.~\ref{fig:1IGT_2D_sig-020_kappa003125}).
 With positive surface charge, however, the tilt angle shifts in such a way that the antigen-binding fragments are pointing outwards of the surface, in a favorable ``tail-on'' position (Fig.~\ref{fig:1IGT_2D_sig020_kappa003125}). This is the combined effect of a higher positive surface charge and the lower shielding effect caused by the reduced salt content, which at the same time increase the local electrostatic interaction with the surface, resulting in a favorable orientation.

From the results in Figure \ref{fig:1IGT_negcharge} and Figure \ref{fig:1IGT_poscharge}, we can conclude that it is easier to control the antibody orientation with low salt concentration and high surface charge. 
In our results, Figures \ref{fig:1IGT_3D_sig-02_kap003125_til124-rot140} and \ref{fig:1IGT_3D_sig02_kap003125_til076-rot160} show the orientation of the antibody at the lowest salt concentration and higher surface charge, but only the orientation in Figure \ref{fig:1IGT_3D_sig02_kap003125_til076-rot160}, with positive surface charge, is favorable for biosensing applications, since the Fab fragments are pointing up.

That favorable orientations for biosensing applications are best obtained with high positive surface charge and low salt concentration is consistent with experimental observations by Chen and co-workers. \cite{ChenLiuZhouJiang2003} These researchers developed a coarse-grained method known as the united residue model,\cite{ZhouChenJiang2003} which qualitatively aligns with our results.
 
