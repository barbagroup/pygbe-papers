%!TEX root = CooperBarba-orientation.tex

The results detailed in this section were obtained using an extension of the open-source code \pygbe,\footnote{\href{https://github.com/barbagroup/pygbe}{https://github.com/barbagroup/pygbe}} accounting for the presence of surfaces with imposed charge or potential.\cite{CooperBarba2015a}
We ran the calculations for protein \gb on a workstation with Intel Xeon X5650 \cpu s  and one \nvidia Tesla C2075 \gpu\ card (late 2011). 
The final case considers the antibody immunoglobulin G, which is a much larger molecule than protein \gb. For these runs, we used Boston University's \textsc{bungee} cluster, which has 16 nodes with 8 Intel Xeon \cpu\ cores each, and a total of 3 \nvidia Tesla Kepler K20 (late 2012) and 26 \nvidia Tesla M2070/2075 \gpu s. All runs were serial: single-\cpu\ and single-\gpu. 
We obtained the van der Waals radii and charge distribution using \texttt{pdb2pqr}\cite{Dolinsky04} with an \amber forcefield, and generated the meshes using the free \msms software.\cite{SannerOlsonSpehner1995}
In these tests, we did not consider a Stern layer for either the protein or the charged surface, nor the presence of solvent-filled cavities inside the protein.

\subsection{First case: protein G\,B1\,D4$^{\prime}$} \label{sec:PGB}

We investigated the preferred orientation of protein \gb placed 2\AA\ away from a 100\AA$\times$100\AA$\times$10\AA\ block with surface charge density $\pm$0.05C/m$^2$, centered with respect to a 100\AA$\times$100\AA\ face.
In biosensors, protein \gb can be used as an intermediate protein, coupled to the functionalized surface directly by covalent bonding.
The protein will thus be at a small distance from the surface. In this case, 2\AA\ is in the order of magnitude of the size of a water molecule, or of a C--N bond.
The change density of $\pm$0.05C/m$^2$ matches that used in other works.\cite{LiuLiaoZhou2013}
As seen in Figure \ref{fig:1pgb_orientation}, $\alpha_\text{tilt}$ is the angle between the protein's dipole moment and the normal vector to the surface, and $\alpha_\text{rot}$ rotates about the dipole moment. 
When the dipole-moment vector and the normal are aligned ($\alpha_\text{rot}=0$), we define a vector $\mathbf{V}_\text{ref}$ as the shortest distance between the axis normal to the surface that goes through the center of mass, and the atom that is furthest away from it. 
We use $\mathbf{V}_\text{ref}$ as a reference to define the rotation angle $\alpha_\text{rot}$: the angle between $\mathbf{V}_\text{ref}$ and the vector normal to a 100\AA$\times$10\AA\ face.  

In these cases, we considered a solvent with no salt, i.e., $\kappa=0$ (to compare with other published results), and with relative permittivity 80. The region inside the protein had a relative permittivity of 4.

We sampled the total free energy every $\Delta \alpha_{\text{tilt}} = 2^\circ$ of tilt angle and $\Delta \alpha_{\text{rot}} =10^\circ$ of rotation angle, resulting in $3,240$ independent runs.  The surface mesh had 4 triangles per square Angstrom on the protein geometry and 2 triangles per square Angstrom on the charged surface. 

Numerical parameters are presented in Table \ref{table:params3}. In a companion publication,\cite{CooperBarba2015a} we present a grid-convergence study using both an analytical solution and a case with protein \gb.\cite{CooperBarba2015-share1348803} We computed an approximate exact value of $-222.43$[kcal/mol] for solvation energy and $317.98$[kcal/mol] for surface energy using Richardson extrapolation with very fine parameters. With results that are less than 2\% away from the approximate exact values, we are comfortable with the parameters in Table \ref{table:params3} and mesh densities of $4$ elements per square Angstrom on the protein and $2$ elements per square Angstrom on the surface.\cite{CooperBarba2015a}

\begin{table}[h]
  %\centering
   %\fontfamily{ppl}\selectfont
   \caption{\label{table:params3}Numerical parameters used for numerically probing the orientation of protein \gb. } 
    \begin{tabular}{c c c c c c c}
	\hline%\toprule
	\multicolumn{3}{l} {\# Gauss points:} & \multicolumn{3}{l}{Treecode:} & \gmres:\\
	\footnotesize{in-element} & \footnotesize{close-by} & \footnotesize{far-away} & $N_{\text{crit}}$ & $P$ &  $\theta$  & tol.\\
	\hline%\midrule
	9 per side & 19 & 1  &  300 & 4 & 0.5  & $10^{-5}$\\	
	\hline%\bottomrule
    \end{tabular}
\end{table}

Using total free energy as the input, the integrals of Equation \eqref{eq:prob_angle} can be computed by means of the trapezoidal rule. Figure \ref{fig:1PGB_probability} presents the probability of the protein orientation in terms of $\cos(\alpha_{\text{tilt}})$, in intervals of $\Delta \cos(\alpha_{\text{tilt}}) = 0.005$ (Fig.~\ref{fig:1PGB_cos}) and $\Delta \alpha_{\text{tilt}}$=2$^{\circ}$ (Fig.~\ref{fig:1PGB_alpha}). Table \ref{table:avg} presents the average orientation $<\cos(\alpha_{\text{tilt}})>$ for the surface having either positive or negative charge density, and Figure \ref{fig:field} shows the electrostatic potential for the preferred orientation in each case. 

\begin{figure*}
   \centering
   \subfloat[]{\includegraphics[width=0.43\textwidth]{Figure11a.pdf} \label{fig:1PGB_cos}}
   \subfloat[]{\includegraphics[width=0.43\textwidth]{Figure11b.pdf} \label{fig:1PGB_alpha}}\\
   \subfloat[]{\includegraphics[width=0.40\textwidth]{Figure11c.pdf} \label{fig:1PGB_2D_sig005}}
   \subfloat[]{\includegraphics[width=0.40\textwidth]{Figure11d.pdf} \label{fig:1PGB_2D_sig-005}}
   \caption{Orientation probability distribution of protein \gb. Figures \ref{fig:1PGB_cos} and \ref{fig:1PGB_alpha} are the probability with respect to the tilt angle and its cosine, respectively. Figures \ref{fig:1PGB_2D_sig005} and \ref{fig:1PGB_2D_sig-005} are the probability distributions with respect to both the tilt and rotation angles. Data sets, figure files and running/plotting scripts are available under \ccby.\cite{CooperBarba2015-share1348804}}
   \label{fig:1PGB_probability}
\end{figure*}

\begin{table}[h]
   \caption{\label{table:avg}Average orientation.} 
    \begin{tabular}{c c}
	\hline
	\multicolumn{2}{c} {$<\cos(\alpha_{\text{tilt}})>$} \\
	Negative & Positive \\
	\hline
	$-0.968$ & $0.963$ \\	
	\hline
    \end{tabular}
\end{table}

\begin{figure*}
   \centering
   \subfloat[Negative surface charge ($\alpha_\text{tilt}=172^\circ$, $\alpha_\text{rot}=110^\circ$)]{\includegraphics[width=0.48\textwidth]{Figure12a.pdf} \label{fig:phi_sig-0.05}} 
   %\subfloat[Surface charge density with negative surface charge]{\includegraphics[width=0.5\textwidth]{dphi_sig-005.pdf} \label{fig:dphi_sig-0.05}} \\
   \subfloat[Positive surface charge ($\alpha_\text{tilt}=8^\circ$, $\alpha_\text{rot}=150^\circ$)]{\includegraphics[width=0.48\textwidth]{Figure12b.pdf} \label{fig:phi_sig0.05}} 
   %\subfloat[Surface charge density with positive surface charge]{\includegraphics[width=0.5\textwidth]{dphi_sig005.pdf} \label{fig:dphi_sig0.05}}
   \caption{Electrostatic potential of protein \gb for the preferred orientations according to Figure \ref{fig:1PGB_probability}. Black arrow indicates direction of dipole-moment vector.}
   \label{fig:field}
\end{figure*}

\subsection{Second case: immunoglobulin G} \label{sec:IGT}


We computed the electrostatic field of immunoglobulin G---a protein widely used in biosensors---interacting with a 250\AA$\times$250\AA$\times$10\AA\ block, varying the conditions of surface charge and salt concentration. The protein was centered with respect to a  250\AA$\times$250\AA\ face, at a distance 5\AA\ above it. The solvent had relative permittivity of 80 and the protein of 4.

\medskip

 \paragraph*{Grid-convergence study for immunoglobulin G---}

Since this was the first time we performed calculations on immunoglobulin G, we carried out a grid-convergence study to make sure the geometry was well resolved and to find adequate values of the simulation parameters for sampling different orientations. The error plotted in Figure \ref{fig:1IGT_convergence} is the relative difference between the energy obtained using \pygbe with each mesh density and the estimated exact value computed with Richardson extrapolation.

In this case, we computed the solvation energy and surface energy of a system consisting of a surface with charge density 0.05C/m$^2$ and a protein with $\alpha_{\text{tilt}} = 31^{\circ}$ and $\alpha_{\text{rot}} = 130^{\circ}$. Using the results from runs with a mesh density of 2, 4, and 8 elements per square Angstrom, we added the solvation and surface energies, and used Richardson extrapolation to obtain a value of $-2792.22$[kcal/mol], and an \emph{observed order of convergence} of 0.85. This is our reference to calculate the errors in Figure \ref{fig:1IGT_convergence}. There is a slight deviation from the expected value of the observed order of convergence (1.0), which we attribute to the non-uniform mesh generated by \msms. Even though the mesh density is on average doubled for each run, there is no guarantee that the refinement is homogeneous throughout the whole molecular surface. The numerical parameters are presented in Table \ref{table:params4}.

\begin{table}[h]
  %\centering
   %\fontfamily{ppl}\selectfont
   \caption{\label{table:params4}Numerical parameters used in the grid-convergence study with immunoglobulin G. } 
    \begin{tabular}{c c c c c c c}
	\hline%\toprule
	\multicolumn{3}{l} {\# Gauss points:} & \multicolumn{3}{l}{Treecode:} & \gmres:\\
	\footnotesize{in-element} & \footnotesize{close-by} & \footnotesize{far-away} & $N_{\text{crit}}$ & $P$ &  $\theta$  & tol.\\
	\hline%\midrule
	9 per side & 19 & 1  &  1000 & 6 & 0.5  & $10^{-5}$\\	
	\hline%\bottomrule
    \end{tabular}
\end{table}




\begin{figure}[h] %  figure placement: here, top, bottom, or page
   \centering
   \includegraphics[width=0.45\textwidth]{Figure13.pdf} 
   \caption{Grid-convergence study of the solvation plus surface energy for immunoglobulin G interacting with a surface with charge density of 0.05C/m$^2$. Data sets, figure files and plotting scripts are available under \ccby.\cite{CooperBarba2015-share1348801}}
   \label{fig:1IGT_convergence}
\end{figure}



\medskip 

 \paragraph*{Probing orientation of immunoglobulin G---}

We sampled the total free energy every $\Delta \alpha_{\text{tilt}} = 4^\circ$ of tilt angle and $\Delta \alpha_{\text{rot}} =20^\circ$ of rotation angle, resulting in a total of 810 runs.  The surface meshes had 2 triangles per square Angstrom throughout. Numerical parameters are presented in Table \ref{table:params5}.

\begin{table}[h]
  %\centering
   %\fontfamily{ppl}\selectfont
   \caption{\label{table:params5}Numerical parameters used in the runs probing orientation of immunoglobulin G. } 
    \begin{tabular}{c c c c c c c}
	\hline%\toprule
	\multicolumn{3}{l} {\# Gauss points:} & \multicolumn{3}{l}{Treecode:} & \gmres:\\
	\footnotesize{in-element} & \footnotesize{close-by} & \footnotesize{far-away} & $N_{\text{crit}}$ & $P$ &  $\theta$  & tol.\\
	\hline%\midrule
	9 per side & 19 & 1  &  300 & 2 & 0.5  & $10^{-4}$\\	
	\hline%\bottomrule
    \end{tabular}
\end{table}

With the computed total free energy, we obtained the probability of each orientation using Equation \eqref{eq:prob_angle} and the trapezoidal rule. %Figure \ref{fig:probability} shows the probability of the protein orientation in terms of $\cos(\alpha_{\text{tilt}})$, in intervals of $\Delta \cos(\alpha_{\text{tilt}}) = 0.05$ for Figure \ref{fig:1IGT_cos}, and $\Delta \alpha_{\text{tilt}}$=4$^{\circ}$ for Figure \ref{fig:1IGT_alpha}.  
We sampled all combinations with surface charges of $\sigma=\pm$0.05C/m$^2$ and $\sigma = \pm$ 0.1C/m$^2$ and salt concentrations of 145mM ($\kappa$ = 0.125 \AA$^{-1}$) and 37mM ($\kappa$ = 0.0625 \AA$^{-1}$). For each of these cases, Figures \ref{fig:1IGT_negcharge}  and \ref{fig:1IGT_poscharge} show a color plot of the probability distribution with respect to the tilt and rotation angles, and a 3D plot of the preferred orientation, where the solvent-excluded surface is colored by the electrostatic potential.

\begin{figure*}
   \centering
   \subfloat[Probability for $\sigma=-0.05$C/m$^2$, $\kappa=0.125$\AA$^{-1}$]{\includegraphics[width=0.4\textwidth]{Figure14a.pdf} \label{fig:1IGT_2D_sig-005}}
   \subfloat[x-y plane view for $\alpha_{\text{tilt}} = 116^{\circ}$ and $\alpha_{\text{rot}} = 100^{\circ}$]{\includegraphics[width=0.4\textwidth]{Figure14b.pdf} \label{fig:1IGT_3D_sig-005_kap0125_til116-rot100}}\\
   \subfloat[Probability for $\sigma=-0.1$C/m$^2$, $\kappa=0.125$\AA$^{-1}$]{\includegraphics[width=0.4\textwidth]{Figure14c.pdf} \label{fig:1IGT_2D_sig-020_kappa01250}}
   \subfloat[y-z plane view for $\alpha_{\text{tilt}} = 36^{\circ}$ and $\alpha_{\text{rot}} = 300^{\circ}$]{\includegraphics[width=0.4\textwidth]{Figure14d.pdf} \label{fig:1IGT_3D_sig-02_kap0125_til056-rot040}}\\
   \subfloat[Probability for $\sigma=-0.05$C/m$^2$, $\kappa=0.0625$\AA$^{-1}$]{\includegraphics[width=0.4\textwidth]{Figure14e.pdf} \label{fig:1IGT_2D_sig-005_kappa003125}}
   \subfloat[x-y plane view for $\alpha_{\text{tilt}} = 40^{\circ}$ and $\alpha_{\text{rot}} = 340^{\circ}$]{\includegraphics[width=0.4\textwidth]{Figure14f.pdf} \label{fig:1IGT_3D_sig-005_kap003125_til116-rot160}}\\
   \subfloat[Probability for $\sigma=-0.1$C/m$^2$, $\kappa=0.0625$\AA$^{-1}$]{\includegraphics[width=0.4\textwidth]{Figure14g.pdf} \label{fig:1IGT_2D_sig-020_kappa003125}}
   \subfloat[x-y plane view for $\alpha_{\text{tilt}} = 40^{\circ}$ and $\alpha_{\text{rot}} = 40^{\circ}$]{\includegraphics[width=0.4\textwidth]{Figure14h.pdf} \label{fig:1IGT_3D_sig-02_kap003125_til124-rot140}}
   \caption{Orientation probability distribution and surface potential of the preferred orientation for immunoglobulin G near a negative surface charge. The black arrow indicates the direction of the dipole moment, and the circles enclose the Fab fragments. Data sets, figure files and plotting scripts available under \ccby.\cite{CooperBarba2015-share1348801}}
   \label{fig:1IGT_negcharge}
\end{figure*}


\begin{figure*}
   \centering
   \subfloat[Probability for $\sigma$=0.05C/m$^2$ and $\kappa$=0.125\AA$^{-1}$]{\includegraphics[width=0.4\textwidth]{Figure15a.pdf} \label{fig:1IGT_2D_sig005}}
   \subfloat[x-y plane view for $\alpha_{\text{tilt}} = 64^{\circ}$ and $\alpha_{\text{rot}} = 280^{\circ}$]{\includegraphics[width=0.4\textwidth]{Figure15b.pdf} \label{fig:1IGT_3D_sig005_kap0125_til064-rot280}}\\
   \subfloat[Probability for $\sigma$=0.1C/m$^2$ and $\kappa$=0.125\AA$^{-1}$]{\includegraphics[width=0.4\textwidth]{Figure15c.pdf} \label{fig:1IGT_2D_sig020_kappa01250}}
   \subfloat[y-z plane view for $\alpha_{\text{tilt}} = 32^{\circ}$ and $\alpha_{\text{rot}} = 100^{\circ}$]{\includegraphics[width=0.4\textwidth]{Figure15d.pdf} \label{fig:1IGT_3D_sig02_kap0125_til064-rot260}}\\
   \subfloat[Probability for $\sigma$=0.05C/m$^2$ and $\kappa$=0.0625\AA$^{-1}$]{\includegraphics[width=0.4\textwidth]{Figure15e.pdf} \label{fig:1IGT_2D_sig005_kappa003125}}
   \subfloat[y-z plane view for $\alpha_{\text{tilt}} = 44^{\circ}$ and $\alpha_{\text{rot}} = 120^{\circ}$]{\includegraphics[width=0.4\textwidth]{Figure15f.pdf} \label{fig:1IGT_3D_sig005_kap003125_til044-rot120}}\\
   \subfloat[Probability for $\sigma$=0.1C/m$^2$ and $\kappa$=0.0625\AA$^{-1}$]{\includegraphics[width=0.4\textwidth]{Figure15g.pdf} \label{fig:1IGT_2D_sig020_kappa003125}}
   \subfloat[x-y plane view for $\alpha_{\text{tilt}} = 64^{\circ}$ and $\alpha_{\text{rot}} = 260^{\circ}$]{\includegraphics[width=0.4\textwidth]{Figure15h.pdf} \label{fig:1IGT_3D_sig02_kap003125_til076-rot160}}
   \caption{Orientation probability distribution and surface potential of the preferred orientation for immunoglobulin G near a positive surface charge. The black arrow indicates the direction of the dipole moment, and the circles enclose the Fab fragments. Data sets, figure files and plotting scripts available under \ccby.\cite{CooperBarba2015-share1348801}}
   \label{fig:1IGT_poscharge}
\end{figure*}


%%% COMMENTED OUT SECTION
\begin{comment}
\begin{figure*}
   \centering
   \subfloat[]{\includegraphics[width=0.45\textwidth]{1IGT_cos.pdf} \label{fig:1IGT_cos}}
   \subfloat[]{\includegraphics[width=0.45\textwidth]{1IGT_alpha.pdf} \label{fig:1IGT_alpha}}\\
   \subfloat[]{\includegraphics[width=0.45\textwidth]{1IGT_2D_sig005.pdf} \label{fig:1IGT_2D_sig005}}
   \subfloat[]{\includegraphics[width=0.45\textwidth]{1IGT_2D_sig-005.pdf} \label{fig:1IGT_2D_sig-005}}
   \caption{Orientation distribution of immunoglobulin G. Figures \ref{fig:1IGT_cos} and \ref{fig:1IGT_alpha} are the probability with respect to the tilt angle and its cosine, respectively. Figures \ref{fig:1IGT_2D_sig005} and \ref{fig:1IGT_2D_sig-005} are the orientation with respect to both the tilt and rotation angle.}
   \label{fig:1IGT_probability}
\end{figure*}

\begin{figure*}
   \centering
   \subfloat[Side view for $\alpha_{\text{tilt}} = 116^{\circ}$ and $\alpha_{\text{rot}} = 100^{\circ}$]{\includegraphics[width=0.45\textwidth]{1IGT_3D_sig-005_til116_rot100_side.pdf} \label{fig:1IGT_sig-005_side}}
   \subfloat[Top view for $\alpha_{\text{tilt}} = 116^{\circ}$ and $\alpha_{\text{rot}} = 100^{\circ}$]{\includegraphics[width=0.45\textwidth]{1IGT_3D_sig-005_til116_rot100_top.pdf} \label{fig:1IGT_sig-005_top}}\\
   \subfloat[Side view for $\alpha_{\text{tilt}} = 64^{\circ}$ and $\alpha_{\text{rot}} = 280^{\circ}$]{\includegraphics[width=0.45\textwidth]{1IGT_3D_sig005_til064_rot280_side.pdf} \label{fig:1IGT_sig005_side}}
   \subfloat[Top view for $\alpha_{\text{tilt}} = 64^{\circ}$ and $\alpha_{\text{rot}} = 280^{\circ}$]{\includegraphics[width=0.45\textwidth]{1IGT_3D_sig005_til064_rot280_top.pdf} \label{fig:1IGT_sig005_top}}
   \caption{Surface potential for Immunoglobulin G for the preferred orientation according to Figure  \ref{fig:1IGT_probability} for $\sigma$=-0.05 C/m$^2$ in Figures \ref{fig:1IGT_sig-005_side} and \ref{fig:1IGT_sig-005_top} and $\sigma$=0.05 C/m$^2$ in Figures \ref{fig:1IGT_sig005_side} and \ref{fig:1IGT_sig005_top}. The black arrow indicates the direction of the dipole moment.}
   \label{fig:1IGT_3D}
\end{figure*}
\end{comment}
%%% END COMMENT


%%% COMMENTED OUT SECTION
\begin{comment} 

\subsection{Fc-protein G complex} \label{sec:FCC}

We simulated the electrostatic field of the complex generated by the Fc fragment of immunoglobulin G and protein G B1 interacting with a $250\times10\times250$\AA \xspace block with surface charge density $\pm$0.05C/m$^2$, to find its preferred orientation. The protein was centered with respect to a  $250\times250$\AA \xspace face, 5\AA \xspace above it, and the orientation was defined by $\alpha_{\text{tilt}}$ and $\alpha_{\text{rot}}$, like in Section \ref{sec:PGB}. 

In these cases, we considered a 0.150M of salt in the solvent, i.e. $\kappa=0.125$, with relative permittivity 80. The region inside the protein had a relative permittivity of 4.

 \paragraph*{Mesh refinement study for Fc-protein G complex}

Similar to Section \ref{sec:PGB}, we performed a mesh refinement study to make sure the geometry was well resolved and to find appropriate simulation parameters for sampling different orientations. 

In these runs, we evaluated the solvation energy and surface energy of a system with a surface with charge density 0.05C/m$^2$, and a protein with $\alpha_{\text{tilt}} = 100^{\circ}$ and $\alpha_{\text{rot}} = 45^{\circ}$. Using the results from runs with mesh density 2, 4, and 8 elements per square Angstrom, we added the solvation and surface energies, and extrapolated them using Richardson extrapolation to obtain -1390.59 kcal/mol, with observed order of convergence 1.1. This value is the reference to calculate the errors in Figure \ref{fig:1FCC_convergence}. 

For the mesh refinement study, we used 1 Gauss points per far-away element, 19 Gauss points per close-by element, and 9 Gauss points per triangle side for the singular integral. The treecode had 6 terms in the Taylor expansion, a multipole-acceptance criterion of 0.5, and no more than 1000 boundary elements per box of the lowest level of the tree. Also, the GMRES tolerance was 10$^{-4}$. 

\begin{figure}[h] %  figure placement: here, top, bottom, or page
   \centering
   \includegraphics[width=0.5\textwidth]{1FCC_convergence.pdf} 
   \caption{Mesh convergence study of the solvation plus surface energy for the Fc-protein G complex interacting with a surface with charge density 0.05C/m$^2$.}
   \label{fig:1FCC_convergence}
\end{figure}

Figure \ref{fig:1FCC_convergence} show errors that are decaying as $1/N$ for both the solvation and surface energies, indicating that the geometry is well resolved.

 \paragraph*{Probing orientation of Fc-protein G complex}

We sampled the total free energy every $\Delta \alpha_{\text{tilt}} = 2^\circ$ of tilt angle and $\Delta \alpha_{\text{rot}} =20^\circ$ of rotation angle.  In these runs, meshes had 2 triangles per square angstrom throughout. We used 1 Gauss point per element further away than 2L of the collocation point, 19 Gauss points for close-by elements, and 9 Gauss points per side of the singular element. The treecode used 2 terms in the Taylor expansion, a multipole-acceptance criterion of 0.5 and no more than 300 elements per lowest level box. The GMRES tolerance was 10$^{-4}$. 

The total free energy was the input for numerically computing the integrals in Equation \eqref{eq:prob_angle} with the trapezoidal rule. Figure \ref{fig:probability} shows the probability of the protein orientation in terms of $\cos(\alpha_{\text{tilt}})$, in intervals of $\Delta \cos(\alpha_{\text{tilt}}) = 0.05$ for Figure \ref{fig:1FCC_cos}, and $\Delta \alpha_{\text{tilt}}$=4$^{\circ}$ for Figure \ref{fig:1IGT_alpha}.  

\begin{figure*}
   \centering
   \subfloat[]{\includegraphics[width=0.45\textwidth]{1FCC_cos.pdf} \label{fig:1FCC_cos}}
   \subfloat[]{\includegraphics[width=0.45\textwidth]{1FCC_alpha.pdf} \label{fig:1FCC_alpha}}\\
   \subfloat[]{\includegraphics[width=0.45\textwidth]{1FCC_2D_sig005.pdf} \label{fig:1FCC_2D_sig005}}
   \subfloat[]{\includegraphics[width=0.45\textwidth]{1FCC_2D_sig-005.pdf} \label{fig:1FCC_2D_sig-005}}
   \caption{Orientation distribution of Fc-protein G complex. Figures \ref{fig:1FCC_cos} and \ref{fig:1FCC_alpha} are the probability with respect to the tilt angle and its cosine, respectively. Figures \ref{fig:1FCC_2D_sig005} and \ref{fig:1FCC_2D_sig-005} are the orientation with respect to both the tilt and rotation angle.}
   \label{fig:1FCC_probability}
\end{figure*}

\begin{figure*}
   \centering
      \subfloat[Side view for $\alpha_{\text{tilt}} = 20^{\circ}$ and $\alpha_{\text{rot}} = 160^{\circ}$]{\includegraphics[width=0.49\textwidth]{1FCC_sig005.pdf}\label{fig:1FCC_sig005}} 
   \subfloat[Side view for $\alpha_{\text{tilt}} = 120^{\circ}$ and $\alpha_{\text{rot}} = 20^{\circ}$]{\includegraphics[width=0.49\textwidth]{1FCC_sig-005.pdf} \label{fig:1FCC_sig-005}}
   \caption{Surface potential for Fc-protein G complex in the preferred orientation according to Figure \ref{fig:1FCC_probability} for $\sigma$=0.05 C/m$^2$ in Figure \ref{fig:1FCC_sig005}, and $\sigma$=-0.05 C/m$^2$ in Figure \ref{fig:1FCC_sig-005}. The black arrow indicates the direction of the dipole moment.}
   \label{fig:1IGT_3D}
\end{figure*}

%%% END COMMENT OUT
\end{comment} 
