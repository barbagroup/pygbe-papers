%!TEX root = CooperBarba-orientation.tex

Various studies have revealed the importance of protein orientation in immunoassays. One work suggested that highly oriented antibodies could result in 100$\times$ improvement in the affinity of a biosensor.\cite{TajimaTakaiIshihara2011} Thus, a design goal would be to know how to prepare a surface to control protein orientation. Yet, despite much work, control of protein orientation has not been successful.
This study increases our understanding of how nanosurface properties (charge) and preparation conditions (salt levels) affect protein orientation.
Our work is the first to successfully have used an implicit-solvent model to study protein orientation near charged surfaces, as far as we know. In a companion publication,\cite{CooperBarba2015a} we describe expanding the applicability of our open-source code, \pygbe, to account for the presence of charged surfaces and present grid-convergence studies using an analytical solution and protein \gb. 

Protein \gb behaves like a point dipole near a charged surface, with the dipole-moment vector shifting $\sim$180$^\circ$ when the sign of the surface charge flips. Our results compare well with experimental observations and simulations using combined Monte Carlo and molecular dynamics methods, supporting the use of our approach for probing protein orientation near charged surfaces.
We applied our approach to immunoglobulin G, a biomolecule that is much larger than protein \gb (about $125\times$, by volume) and would be challenging  to study via molecular dynamics. 
The iso-type \ig 2a was found by previous studies to be hard to control, exhibiting many orientations, but we are able to obtain a high-probability favorable orientation with a positive surface charge of 0.02C/m$^{2}$ and 9mM of salt in the solvent. We conclude that local electrostatic interactions dominate over the dipole moment, and even this protein can be favorably oriented with the appropriate fabrication protocol. Potentially, protein engineering could be used to obtain ligand molecules that interact with charged surfaces in a desired fashion.
In this application, where ligand molecules undergo little conformational change as they adsorb on the sensor surface, our new implicit-solvent model can offer a valuable approach to assist in biosensor design. In our future work, and in collaboration with experimental researchers, we intend to use this approach to aid the design of better ligand molecules, by looking at the preferred orientations for different ligand mutants. 
