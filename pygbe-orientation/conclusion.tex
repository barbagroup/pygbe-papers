%!TEX root = CooperBarba2014.tex

In this work, we successfully used an implicit-solvent model to study protein orientation near charged surfaces. We present for the first time and apply an extension of our open-source \pygbe code to account for the presence of charged surfaces. The new feature of the code was verified against an analytical solution, which we derived for that purpose. 

Using \pygbe, we obtained that protein G B1 D4$^\prime$ behaves like a point dipole near a charged surface, with the dipole-moment vector shifting $\sim$180$^\circ$ when the sign of the surface charge flips. Our results compare well with experimental observations and simulations using molecular dynamics, supporting the use of our approach for probing protein orientation near charged surfaces.
We applied our approach to immunoglobulin G, a biomolecule that is much larger than protein G B1 (about $125\times$, by volume) and would be challenging  to study via molecular dynamics. Through this study, we realized that this protein is best immobilized on a surface with positive charge, for example with a NH$^+_3$ self-assembled monolayer, using high surface charge and low salt concentration. 

We conclude that this implicit-solvent model can offer a valuable approach in biosensor studies. In this application, ligand molecules undergo little conformational change as they adsorb on the sensor surface, and thus the assumption of a rigid structure is valid. In our future work, we intend to use this approach to aid the design of better ligand molecules, by looking at the orientation for different ligand molecule mutants. 