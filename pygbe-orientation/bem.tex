%!TEX root = CooperBarba-orientation.tex


We apply Green's second identity to the system of partial-differential equations in \eqref{eq:pde}, and evaluate the resulting equations on $\Gamma_1$ and $\Gamma_2$ to obtain the following system of integral equations:
%
\begin{widetext}
\begin{align} \label{eq:integral_eq}
\frac{\phi_{1,\Gamma_1}}{2}+ K_{L}^{\Gamma_1}(\phi_{1,\Gamma_1}) -  V_{L}^{\Gamma_1} \left(\frac{\partial}{\partial \mathbf{n}}\phi_{1,\Gamma_1} \right)  =  
\frac{1}{\epsilon_1} \sum_{k=0}^{N_q} \frac{q_k}{4\pi|\mathbf{r}_{\Gamma_1} - \mathbf{r}_k|} &  \quad \text{on $\Gamma_1$,} \nonumber \\ 
\frac{\phi_{1,\Gamma_1}}{2} - K_{Y}^{\Gamma_1}(\phi_{1,\Gamma_1}) +  \frac{\epsilon_1}{\epsilon_2} V_{Y}^{\Gamma_1} \left( \frac{\partial}{\partial \mathbf{n}} \phi_{1,\Gamma_1} \right) -  
K_{Y}^{\Gamma_1}(\phi_{2,\Gamma_2})  + V_{Y}^{\Gamma_1} \left( -\frac{\sigma_0}{\epsilon_2} \right)  = 0& \quad \text{on $\Gamma_1$,} \nonumber \\ 
- K_{Y}^{\Gamma_2}(\phi_{1,\Gamma_1}) + \frac{\epsilon_1}{\epsilon_2} V_{Y}^{\Gamma_2}  \left( \frac{\partial}{\partial \mathbf{n}} \phi_{1,\Gamma_1} \right) + \frac{\phi_{2,\Gamma_2}}{2} - 
K_{Y}^{\Gamma_2}(\phi_{2,\Gamma_2}) +  V_{Y}^{\Gamma_2} \left( -\frac{\sigma_0}{\epsilon_2} \right)  = 0& \quad \text{on $\Gamma_2$.}
\end{align}
\end{widetext}


\noindent The function $\phi_{i,\Gamma_j} = \phi_i(\mathbf{r}_{\Gamma_j})$ is the electrostatic potential at a point that approaches the surface $\Gamma_j$ from the region $\Omega_i$, and
$K$ and $V$ are known as the single- and double-layer potentials, correspondingly:
%
\begin{align} \label{eq:layers}
K_{L/Y}^{\Gamma_k}(\phi_{i,\Gamma_j}) &= \oint_{\Gamma_j} \frac{\partial}{\partial \mathbf{n}} \left[ G_{L/Y}(\mathbf{r}_{\Gamma_k},\mathbf{r}_{\Gamma_j}) \right]\phi_{i,\Gamma_j} \, \mathrm{d} \Gamma, \nonumber \\
V_{L/Y}^{\Gamma_k} \left( \frac{\partial}{\partial \mathbf{n}} \phi_{i,\Gamma_j} \right) &= \oint_{\Gamma_j} \frac{\partial}{\partial \mathbf{n}} \phi_{i,\Gamma_j} G_{L/Y}(\mathbf{r}_{\Gamma_k},\mathbf{r}_{\Gamma_j})  \, \mathrm{d} \Gamma.
\end{align}

\noindent In Equation \eqref{eq:layers}, $G_L$ and $G_Y$ are the free-space Green's functions of the Poisson and linearized Poisson-Boltzmann equations, respectively. 
The single-layer potential of a distribution $\psi$ on a surface $\Gamma$ evaluated at $\mathbf{r}$, $V^\mathbf{r}(\psi_\Gamma)$, can be interpreted as the potential on $\mathbf{r}$ due to a charge distribution $\psi$ on $\Gamma$. 
Similarly, $K^\mathbf{r}(\psi_\Gamma)$ can be seen as the potential induced by a double layer of charges ($\psi$) with opposite sign at $\Gamma$.

Rearranging terms, we write Equation \eqref{eq:integral_eq} in matrix form, as follows:
%
 \begin{align} \label{eq:matrix_dphi}
 \left[
    \begin{matrix} % or pmatrix or bmatrix or Bmatrix or ...
       \frac{1}{2} + K_{L}^{\Gamma_1} & -V_{L}^{\Gamma_1} & 0 \vspace{0.2cm}\\
       \frac{1}{2} - K_{Y}^{\Gamma_1} &  \frac{\epsilon_1}{\epsilon_2} V_{Y}^{\Gamma_1} & -K_{Y}^{\Gamma_1} \vspace{0.2cm} \\
       - K_{Y}^{\Gamma_2} & \frac{\epsilon_1}{\epsilon_2} V_{Y}^{\Gamma_2} & \left(\frac{1}{2} - K_{Y}^{\Gamma_2}\right) \\
    \end{matrix}
    \right] \left[ 
    \begin{matrix} % or pmatrix or bmatrix or Bmatrix or ...
       \phi_{1,\Gamma_1} \vspace{0.2cm} \\
       \frac{\partial}{\partial \mathbf{n}} \phi_{1,\Gamma_1} \vspace{0.2cm}\\
       \phi_{2,\Gamma_2}\\
    \end{matrix} 
     \right] =   \nonumber \\
    \left[
    \begin{matrix} % or pmatrix or bmatrix or Bmatrix or ...
       \sum_{k=0}^{N_q} \frac{q_k}{4\pi|\mathbf{r}_{\Gamma_1} - \mathbf{r}_k|} \vspace{0.2cm} \\
        V_{Y}^{\Gamma_1} \left( \frac{\sigma_0}{\epsilon_2} \right) \vspace{0.2cm} \\
        V_{Y}^{\Gamma_2} \left( \frac{\sigma_0}{\epsilon_2} \right)
    \end{matrix}
    \right].
 \end{align}

The boundary-integral formulation is not limited to represent the protein with a single surface, but can account for solvent-filled cavities inside the protein region and Stern layers.\cite{CooperBardhanBarba2013} In those cases, more than one surface is required to appropriately represent the protein. 
Our implementation follows the guidelines from Altman and co-workers\cite{AltmanBardhanWhiteTidor09} to deal with multiple surfaces.

The boundary-integral formulation of the implicit-solvent model is a popular alternative to compute solvation energies of proteins,\cite{YoonLenhoff1990, Juffer1991a, LuETal2006, BajajETal2011, AltmanBardhanWhiteTidor09, GengKrasny2013, CooperBardhanBarba2013} but the effect of charged surfaces has rarely been considered. The only work that we know of that does include these effects is limited to plane, infinite surfaces.\cite{YoonLenhoff1992} 
