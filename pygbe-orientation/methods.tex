
%!TEX root = CooperBarba2014.tex

\subsection{Discretization and implementation details}

We solve the system in \eqref{eq:integral_eq} numerically using a boundary element method (\bem). In it, we discretize the \ses in flat triangular panels where the potential $(\phi)$ and its normal derivative $(\partial \phi /\partial \mathbf{n})$ are constant, and then collocate the discretized equation on the center of each panel. This transforms the integral operators in the matrix equation \eqref{eq:matrix_dphi} into block matrices of size $N_p \times N_p$, where $N_p$ is the number of panels. Each entry of the block matrix is an integral over one panel ($\Gamma_j$), evaluated on the center of panel $\Gamma_i$:
%
\begin{align} \label{eq:layers_element}
K_{L,ij} &= \int_{\Gamma_j} \frac{\partial}{\partial \mathbf{n}} \left[ G_L(\mathbf{r}_{\Gamma_i},\mathbf{r}_{\Gamma_j}) \right]\mathrm{d} \Gamma_j, \nonumber \\
V_{L,ij} &= \int_{\Gamma_j} G_L(\mathbf{r}_{\Gamma_i},\mathbf{r}_{\Gamma_j})  \mathrm{d} \Gamma_j.
\end{align}

We classify the integrals in Equation \eqref{eq:layers_element} in three, depending on the distance between the panel and the collocation point. 
When the collocation point is inside the panel being integrated, we get a singular integral which we solve with a semi-analytical approach\cite{ZhuHuangSongWhite2001} that places Gauss nodes on the sides of the triangle. 
Integrals where the distance is less than $2L$ ---where $L = \sqrt{2\cdot \text{Area}}$--- are near-singular, and we use a high order Gauss quadrature rule with 19 or more nodes. 
Finally, when the panel and collocation point are further than $2L$, we only need 1, 3, 4 or 7 Gauss nodes per element to get good accuracy.

We solve the resulting linear system with a general minimal residual method (\gmres). The most time consuming part of the \gmres is a matrix-vector multiplication (an $O(N^2)$ operation) done every iteration of the solver, which we approximate in $O(N\log N)$ time with a treecode algorithm\cite{BarnesHut1986}. More details on our implementation of the \bem can be found in our earlier work,\cite{CooperBarba-share154331} and a companion paper.\cite{CooperBarba2015a}
