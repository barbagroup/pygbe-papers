%% Use the option review to obtain double line spacing 
%% \documentclass[preprint,review,12pt]{elsarticle} 
%% Use the options 1p,twocolumn; 3p; 3p,twocolumn; 5p; or 5p,twocolumn 
%% for a journal layout: 
%% \documentclass[final,1p,times]{elsarticle} 
%% \documentclass[final,1p,times,twocolumn]{elsarticle} 
%% \documentclass[final,3p,times]{elsarticle} 
%%\documentclass[final,3p,times,twocolumn]{elsarticle} 
%% \documentclass[final,5p,times]{elsarticle} 
 \documentclass[final,5p,times,twocolumn]{elsarticle} 

\usepackage{amsfonts}
\usepackage{amsmath}
\usepackage{amssymb}
\usepackage{booktabs}
\usepackage{caption}
\usepackage{color}
\usepackage{comment}
\usepackage{graphicx}
\usepackage{hyperref}
\usepackage[utf8]{inputenc} % allows using accents directly in text, like ÔøΩ
\usepackage{subfig}
\usepackage{xspace}


\captionsetup{justification=raggedright,
singlelinecheck=false
}

\newcommand{\pygbe}{\texttt{PyGBe}\xspace}
\newcommand{\gmres}{\textsc{gmres}\xspace}
\newcommand{\bem}{\textsc{bem}\xspace}
\newcommand{\ses}{\textsc{ses}\xspace}
\newcommand{\sam}{\textsc{sam}}
\newcommand{\gpu}{\textsc{gpu}}
\newcommand{\cpu}{\textsc{cpu}}
\newcommand{\apbs}{\textsc{apbs}\xspace}
\newcommand{\nvidia}{\textsc{nvidia}\xspace}
\newcommand{\msms}{\texttt{\textsc{msms}}\xspace}
\newcommand{\amber}{\texttt{\textsc{amber}}\xspace}
\newcommand{\ccby}{\textsc{cc-by}\xspace}

\graphicspath{{figs/}} %  PATH to figure files-- change to ./ for submission

%% The lineno packages adds line numbers. Start line numbering with
%% \begin{linenumbers}, end it with \end{linenumbers}. Or switch it on
%% for the whole article with \linenumbers after \end{frontmatter}.
%% \usepackage{lineno}

%% natbib.sty is loaded by default. However, natbib options can be
%% provided with \biboptions{...} command. Following options are
%% valid:

%%   round  -  round parentheses are used (default)
%%   square -  square brackets are used   [option]
%%   curly  -  curly braces are used      {option}
%%   angle  -  angle brackets are used    <option>
%%   semicolon  -  multiple citations separated by semi-colon
%%   colon  - same as semicolon, an earlier confusion
%%   comma  -  separated by comma
%%   numbers-  selects numerical citations
%%   super  -  numerical citations as superscripts
%%   sort   -  sorts multiple citations according to order in ref. list
%%   sort&compress   -  like sort, but also compresses numerical citations
%%   compress - compresses without sorting
%%
%% \biboptions{comma,round}

% \biboptions{}


\journal{Computer Physics Communications}

\begin{document}

\begin{frontmatter}

%% Title, authors and addresses

%% use the tnoteref command within \title for footnotes;
%% use the tnotetext command for the associated footnote;
%% use the fnref command within \author or \address for footnotes;
%% use the fntext command for the associated footnote;
%% use the corref command within \author for corresponding author footnotes;
%% use the cortext command for the associated footnote;
%% use the ead command for the email address,
%% and the form \ead[url] for the home page:
%%
%% \title{Title\tnoteref{label1}}
%% \tnotetext[label1]{}
%% \author{Name\corref{cor1}\fnref{label2}}
%% \ead{email address}
%% \ead[url]{home page}
%% \fntext[label2]{}
%% \cortext[cor1]{}
%% \address{Address\fnref{label3}}
%% \fntext[label3]{}

\title{Poisson-Boltzmann model for protein-surface electrostatic interactions and grid-convergence study using the \pygbe code}

\author[bu,usm]{Christopher D. Cooper\corref{cdc}}
\ead{cdcooper@bu.edu}

\author[gwu]{Lorena A.~Barba\corref{lab}}
\ead{labarba@gwu.edu}

\address[bu]{Department of Mechanical Engineering, Boston University, Boston, MA.}
\address[gwu]{Department of Mechanical \& Aerospace Engineering, The George Washington University, Washington, D.C.}
\address[usm]{Department of Mechanical Engineering, Universidad T\'ecnica Federico Santa Mar\'ia, Valpara\'iso, Chile.}
\cortext[lab]{\href{mailto:labarba@gwu.edu}{labarba@gwu.edu}}

%\date{\today}

\begin{abstract}


The aim of this work is developing and assessing a computational model to study proteins interacting with charged surfaces and obtain orientation data. After extending the implicit-solvent model used in the open-source code \pygbe\ and deriving an analytical solution for simple geometry, our careful grid-convergence analysis builds confidence on the correctness and value of our approach for probing protein orientation. 


\end{abstract}

\begin{keyword}
biomolecular electrostatics \sep protein surface interaction \sep implicit solvent \sep Poisson-Boltzmann \sep boundary element method \sep treecode \sep Python \sep CUDA

%% MSC codes here, in the form: \MSC code \sep code
%% or \MSC[2008] code \sep code (2000 is the default)

\end{keyword}

\end{frontmatter}

% Body of paper.

\section{Introduction}\label{sec:intro}
%!TEX root = CooperBarba2014.tex

Protein adsorption plays an important role in many biotechnological applications, such as tissue engineering, biomedical implants and biosensors.
Yet, despite their importance, a full understanding of the specific mechanisms remains elusive.\cite{Gray2004,RabeVerdesSeegel2011}

In the field of biosensors, protein adsorption needs to be engineered to obtain a successful device. 
Biosensors detect specific molecules using a nanoscale sensing element, like a metallic nanoparticle or nanowire covered with a bioactive coating. 
The prevalent way to modify a sensor surface is via a self-assembled monolayer (\sam) of a small charged group, with ligand molecules layered on top to achieve the desired function. 
Antibodies are a common choice for the ligand molecules, although the newest devices use single-domain or single-chain fragment molecules.\cite{ByunETal2013,TrillingETal2014} 
Sensing occurs when a target biomolecule binds to the ligand molecule,  changing some physical parameter on the sensor, such as current in nanowires or plasmon resonance frequency in metallic nanoparticles. 

One of the factors affecting biosensor performance is the orientation of ligand molecules.\cite{TajimaTakaiIshihara2011,TrillingBeekwilderZuilhof2013} 
These have specific binding sites, which need to be accessible to the target molecule for the biosensor to function well.
Probing protein orientation is thus one key goal of adsorption studies.
The aim of this study is to develop and assess a computational model to simulate proteins near surfaces and obtain orientation data.

We use an implicit-solvent approach based on the Poisson-Boltzmann equation and fixed protein structures. A sensor element, functionalized with the \sam, can be represented as a charged surface that interacts electrostatically with a biomolecule. Ignoring conformational changes of the biomolecule is justified in this application, since binding sites should remain nearly unmodified during the fabrication process.\cite{TajimaTakaiIshihara2011} 

Previous studies on protein-surface interaction using the Poisson-Boltzmann equation have shown that the model is adequate as long as conformational changes in the protein are slight,\cite{YaoLenhoff2004,YaoLenhoff2005} and that for realistic molecular geometries van der Waals effects can be neglected.\cite{RothNealLenhoff1996}
A similar continuum framework has been used in the past for protein orientation studies,\cite{JufferArgosDevlieg1996} however, it includes ions explicitly. Other studies used a coarse-grained model of the molecule, represented as  a set of spheres,\cite{ShengTsaoZhouJiang2002,ZhouTsaoShengJiang2004} assigned effective charges at the residue level,\cite{FreedCramer2011,ZhouChenJiang2003} or made approximations to account for pH effects.\cite{BiesheuvelvanderVeenNord2005,HartvigdeWeertOstergaartJorgensenJensen2011}

We have added the capability of modeling a protein near a charged surface to our code \pygbe , an open-source code\footnote{\url{https://github.com/barbagroup/pygbe}}  that uses \gpu\ hardware.  Previously, we verified and validated \pygbe in its use to obtain solvation and binding energies, by comparing with analytical solutions of the equations and with results obtained using the well-known \apbs software.\cite{CooperBarba-share154331,CooperBardhanBarba2013} 
This extension was recently verified against an analytical solution valid for a spherical molecule interacting with a spherical charged surface.\cite{CooperBarba2015a}
In the present work, we studied two proteins (GB1 D4' and immunoglobulin-G) near charged surfaces to obtain their preferred orientation, and compared them to several other published results.
We anticipate this modeling tool to be useful for understanding the behavior of proteins as they adsorb on \sam s, potentially aiding the design of better ligand molecules for biosensors.



\section{Implicit-solvent model for proteins near charged surfaces} \label{sec:implicit_solvent}
    %!TEX root = CooperBarba2014.tex

The implicit-solvent model uses continuum electrostatics to describe the mean-field potential in a molecular system. A typical system consists of a protein in a solvent, defining two regions: inside and outside the protein, with an interface marked by the solvent-excluded surface (\ses).  The \ses, beyond which a water molecule cannot penetrate into the protein, can be generated by rolling a (virtual) spherical probe of the size of a water molecule around the protein (see Figure \ref{fig:forcefield-ses}). Inside the protein, the domain has low permittivity ($\epsilon= 2\text{ to }4$) and there are point charges located at the positions of the atoms. The solvent region, representing water with salt, has a permittivity of $\epsilon \approx 80$. A system of partial differential equations models this situation, with a Poisson equation governing inside the protein and a linearized Poisson-Boltzmann equation governing in the solvent region. Appropriate interface conditions on the \ses express the continuity of the potential and electric displacement, completing the mathematical formulation.

\begin{figure}% [h]
   \includegraphics[width=0.49\textwidth]{Figure1.pdf} 
   \caption{Sketch of the process for generating a solvent-excluded surface (\ses): a protein molecule contains a set of atoms that define a radius upon applying a force field and a probe the size of a water molecule is rolled to define the  \ses. $\Omega_1$ is the protein region and $\Omega_2$ the solvent region.}
   \label{fig:forcefield-ses}
\end{figure}

This model has been widely applied to investigate interactions between molecules, such as in protein-ligand binding. We are interested here in an extension of the model to consider interactions between proteins and surfaces with an imposed charge. This new setup is sketched in Figure \ref{fig:molecule_surface}, and is described mathematically by the following equations:


\begin{align} \label{eq:pde}
\nabla^2 \phi_1(\mathbf{r}) &= - \sum_k \frac{q_k}{\epsilon_1} \delta(\mathbf{r},\mathbf{r}_k) \ \text{ in solute $(\Omega_1)$,}  \nonumber \\ 
\nabla^2\phi_2 (\mathbf{r}) &= \kappa^2 \phi_2(\mathbf{r}) \quad \qquad \ \ \text{ in solvent $(\Omega_2)$,}  \nonumber \\ 
\phi_1 &=\phi_2 \qquad \qquad \qquad \text{ on interface $\Gamma_1$,}  \nonumber \\ 
\epsilon_1 \frac{\partial \phi_1}{\partial \mathbf{n}} &= \epsilon_2 \frac{\partial \phi_2}{\partial \mathbf{n}} \nonumber \\
-\epsilon_2 \frac{\partial \phi_2}{\partial \mathbf{n}} &= \sigma_0 \qquad \qquad \qquad \text{ on surface $\Gamma_2$,} 
\end{align}

\noindent Here, $\phi_i$ is the potential corresponding to the region $\Omega_i$ with permittivity $\epsilon_i$, and $\sigma_0$ is the set charge on the nanosurface. The surface $\Gamma_2$ could correspond to a device such as a biosensor.

\begin{figure}
   \includegraphics[width=0.45\textwidth]{Figure2.pdf} 
   \caption{Sketch of a molecule interacting with a surface: $\Omega_1$ is the protein, $\Omega_2$ the solvent region, $\Gamma_1$ is the  \ses and $\Gamma_2$ a nanosurface with imposed charge.}
   \label{fig:molecule_surface}
\end{figure}


\paragraph*{Boundary integral formulation} \label{sec:bie}
%!TEX root = CooperBarba2014.tex



We express the system of partial-differential equations in  \eqref{eq:pde} by the corresponding integral equations along the interface and the nanosurface, $\Gamma_1$ and $\Gamma_2$, which gives

\begin{widetext}
\begin{align} \label{eq:integral_eq}
\frac{\phi_{1,\Gamma_1}}{2}+ K_{L}^{\Gamma_1}(\phi_{1,\Gamma_1}) -  V_{L}^{\Gamma_1} \left(\frac{\partial}{\partial \mathbf{n}}\phi_{1,\Gamma_1} \right)  =  
\frac{1}{\epsilon_1} \sum_{k=0}^{N_q} \frac{q_k}{4\pi|\mathbf{r}_{\Gamma_1} - \mathbf{r}_k|} &  \quad \text{on $\Gamma_1$,} \nonumber \\ 
\frac{\phi_{1,\Gamma_1}}{2} - K_{Y}^{\Gamma_1}(\phi_{1,\Gamma_1}) +  \frac{\epsilon_1}{\epsilon_2} V_{Y}^{\Gamma_1} \left( \frac{\partial}{\partial \mathbf{n}} \phi_{1,\Gamma_1} \right) -  
K_{Y}^{\Gamma_1}(\phi_{2,\Gamma_2})  + V_{Y}^{\Gamma_1} \left( -\frac{\sigma_0}{\epsilon_2} \right)  = 0& \quad \text{on $\Gamma_1$,} \nonumber \\ 
- K_{Y}^{\Gamma_2}(\phi_{1,\Gamma_1}) + \frac{\epsilon_1}{\epsilon_2} V_{Y}^{\Gamma_2}  \left( \frac{\partial}{\partial \mathbf{n}} \phi_{1,\Gamma_1} \right) + \frac{\phi_{2,\Gamma_2}}{2} - 
K_{Y}^{\Gamma_2}(\phi_{2,\Gamma_2}) +  V_{Y}^{\Gamma_2} \left( -\frac{\sigma_0}{\epsilon_2} \right)  = 0& \quad \text{on $\Gamma_2$.}
\end{align}
\end{widetext}


\noindent where $\phi_{i,\Gamma_j} = \phi_i(\mathbf{r}_{\Gamma_j})$ is the potential in region $\Omega_i$ evaluated at the surface $\Gamma_j$. $K$ and $V$ are defined as

%
\begin{align} \label{eq:layers}
K_{L/Y}^{\Gamma_k}(\phi_{i,\Gamma_j}) &= \oint_{\Gamma_j} \frac{\partial}{\partial \mathbf{n}} \left[ G_{L/Y}(\mathbf{r}_{\Gamma_k},\mathbf{r}_{\Gamma_j}) \right]\phi_{i,\Gamma_j} \, \mathrm{d} \Gamma, \nonumber \\
V_{L/Y}^{\Gamma_k} \left( \frac{\partial}{\partial \mathbf{n}} \phi_{i,\Gamma_j} \right) &= \oint_{\Gamma_j} \frac{\partial}{\partial \mathbf{n}} \phi_{i,\Gamma_j} G_{L/Y}(\mathbf{r}_{\Gamma_k},\mathbf{r}_{\Gamma_j})  \, \mathrm{d} \Gamma,
\end{align}

\noindent corresponding to the double- and single-layer potentials of $\phi_{i,\Gamma_j}$ and $\frac{\partial}{\partial \mathbf{n}} \phi_{i,\Gamma_j}$ evaluated at a point on the surface $\Gamma_k$. The functions $G_L$ and $G_Y$ are the free-space Green's functions of the Poisson (Laplace kernel) and linearized Poisson-Boltzmann (Yukawa kernel) equations, respectively. Rearranging terms, we write Equation \eqref{eq:integral_eq} in matrix form, as follows:
%
 \begin{align} \label{eq:matrix_dphi}
 \left[
    \begin{matrix} % or pmatrix or bmatrix or Bmatrix or ...
       \frac{1}{2} + K_{L}^{\Gamma_1} & -V_{L}^{\Gamma_1} & 0 \vspace{0.2cm}\\
       \frac{1}{2} - K_{Y}^{\Gamma_1} &  \frac{\epsilon_1}{\epsilon_2} V_{Y}^{\Gamma_1} & -K_{Y}^{\Gamma_1} \vspace{0.2cm} \\
       - K_{Y}^{\Gamma_2} & \frac{\epsilon_1}{\epsilon_2} V_{Y}^{\Gamma_2} & \left(\frac{1}{2} - K_{Y}^{\Gamma_2}\right) \\
    \end{matrix}
    \right] \left[ 
    \begin{matrix} % or pmatrix or bmatrix or Bmatrix or ...
       \phi_{1,\Gamma_1} \vspace{0.2cm} \\
       \frac{\partial}{\partial \mathbf{n}} \phi_{1,\Gamma_1} \vspace{0.2cm}\\
       \phi_{2,\Gamma_2}\\
    \end{matrix} 
     \right] =   \nonumber \\
    \left[
    \begin{matrix} % or pmatrix or bmatrix or Bmatrix or ...
       \sum_{k=0}^{N_q} \frac{q_k}{4\pi|\mathbf{r}_{\Gamma_1} - \mathbf{r}_k|} \vspace{0.2cm} \\
        V_{Y}^{\Gamma_1} \left( \frac{\sigma_0}{\epsilon_2} \right) \vspace{0.2cm} \\
        V_{Y}^{\Gamma_2} \left( \frac{\sigma_0}{\epsilon_2} \right)
    \end{matrix}
    \right].
 \end{align}

The boundary-integral formulation is not limited to represent the protein with a single surface, but can account for solvent-filled cavities inside the protein region and Stern layers,\cite{CooperBardhanBarba2013} where more surfaces are required for an appropriate model. This implementation follows the guidelines from Altman and co-workers to deal with multiple surfaces.

The boundary-integral formulation of the implicit-solvent model is a popular alternative to compute solvation energies of proteins,\cite{YoonLenhoff1990, Juffer1991a, LuETal2006, BajajETal2011, AltmanBardhanWhiteTidor09, GengKrasny2013, CooperBardhanBarba2013} but little work includes the effect of charged surfaces. Moreover, the efforts that do include them\cite{YoonLenhoff1992} are limited to plane, infinite-sized surfaces. 



%=============
\section{Methods}\label{sec:methods}

%!TEX root = CooperBarba2014.tex

\subsection{Discretization}

To numerically solve the system in \eqref{eq:matrix_dphi}, we discretize the boundaries into flat triangular panels and assume that $\phi$ and $\frac{\partial \phi}{\partial \mathbf{n}}$ are constant within those panels. The discretized form of the integral operators is as follows:
%
\begin{align} \label{eq:layers_disc}
&K_{L,\text{disc}}^{\mathbf{r}_i}\left(\phi(\mathbf{r}_{\Gamma})\right) =  \sum_{j=1}^{N_p}\phi(\mathbf{r}_{\Gamma_j})\int_{\Gamma_j} \frac{\partial}{\partial \mathbf{n}} \left[ G_L(\mathbf{r}_{i},\mathbf{r}_{\Gamma_j}) \right]\mathrm{d} \Gamma_j,  \nonumber \\
&V_{L,\text{disc}}^{\mathbf{r}_i} \left( \frac{\partial}{\partial \mathbf{n}} \phi(\mathbf{r}_{\Gamma}) \right) = \sum_{j=1}^{N_p} \frac{\partial}{\partial \mathbf{n}} \phi(\mathbf{r}_{\Gamma_j}) \int_{\Gamma_j} G_L(\mathbf{r}_{i},\mathbf{r}_{\Gamma_j})  \mathrm{d} \Gamma_j,
\end{align}

\noindent where $N_p$ is the number of discretization elements on $\Gamma$, and $\phi(\mathbf{r}_{\Gamma_j})$ and $\frac{\partial}{\partial \mathbf{n}} \phi(\mathbf{r}_{\Gamma_j})$ are the constant values of $\phi$ and $\frac{\partial \phi}{\partial \mathbf{n}}$ on panel $\Gamma_j$ (we are somewhat abusing the nomenclature here by reusing the symbol $\Gamma$, which previously referred to the complete surface). By collocating $\mathbf{r}_i$ on the center of each panel, we get a linear system of equations that look just like \eqref{eq:matrix_phi} or \eqref{eq:matrix_dphi}, but the coefficient matrix is formed by sub-matrices of size $N_p \times N_p$ rather than integral operators. Each element of a sub-matrix is an integral over one panel $\Gamma_j$, with $\mathbf{r}_i$ located at the center of the collocation panel $\Gamma_i$, as follows:

\begin{align} \label{eq:layers_element}
K_{L,ij} &= \int_{\Gamma_j} \frac{\partial}{\partial \mathbf{n}} \left[ G_L(\mathbf{r}_{\Gamma_i},\mathbf{r}_{\Gamma_j}) \right]\mathrm{d} \Gamma_j, \nonumber \\
V_{L,ij} &= \int_{\Gamma_j} G_L(\mathbf{r}_{\Gamma_i},\mathbf{r}_{\Gamma_j})  \mathrm{d} \Gamma_j.
\end{align}

The terms on the right-hand side and the unknown vectors in the discretized form of Equation \eqref{eq:matrix_phi} are sub-vectors of size $N_p$. In this case, each element is the evaluation on the collocation panel $\Gamma_i$, written as
%
\begin{align} \label{eq:vector_disc}
\phi_{1,\Gamma_1} &= \phi_1(\mathbf{r}_i), \nonumber \\
\frac{\partial}{\partial \mathbf{n}}\phi_{1,\Gamma_1} &= \frac{\partial}{\partial \mathbf{n}}\phi_1(\mathbf{r}_i), \nonumber \\
\sum_{k=0}^{N_q} \frac{q}{4\pi|\mathbf{r}_{\Gamma_1} - \mathbf{r}_k|} &= \sum_{k=0}^{N_q} \frac{q}{4\pi|\mathbf{r}_i - \mathbf{r}_k|},
\end{align}

\noindent where $\mathbf{r}_i$ is located at the center of panel $\Gamma_i$. The surface is smooth at the center of each panel, hence, we avoid any complications related to non-smooth boundaries in \bem by using centered collocation.


In our numerical solution, integrals are calculated in three possible ways, depending on how close the panel is to the collocation point. When the collocation point is inside the element being integrated, we use a semi-analytical technique, with Gauss points placed along the edges of the element \cite{HessSmith1967,ZhuHuangSongWhite2001}. If the integrated element is closer than $2L$ from the collocation point ---where $L = \sqrt{2\cdot A_j}$ for $A_j$ the area of panel $j$--- we use a fine Gauss quadrature rule, with 19 or more points per element. Beyond a distance of $2L$, elements have only 1, 3, 4 or 7 Gauss points, depending on the case.


\subsection{Treecode-accelerated boundary element method}

Most modern implementations of the boundary element method (\bem) use Krylov methods to solve the linear system, usually a general minimal residual method (\gmres), which is agnostic to the structure of the matrix. In practice, Krylov solvers for \bem require $O(n \cdot N_p^2)$ operations to obtain the unknown vector, where $n$ is the number of iterations to get a desired residual, and is much smaller than $N_p$. The $O(N^2)$ scaling is given by a matrix-vector product (with a dense matrix) done in every iteration; this is the most time-consuming part of the algorithm, and makes \bem prohibitive for more than a few thousand discretization elements. 

But when we inspect the approximation of the integrals in  \eqref{eq:layers_element} with Gauss quadrature rules, we see that the matrix-vector product has the form of an $N$-body problem, similar to gravitational potential calculations in planetary systems. In this case, the Gauss quadrature points act analogously to planets (sources of mass) and the collocation points are analogous to the locations where the gravitational potential is computed (targets points). There are several ways to accelerate this kind of computations, for example fast-multipole methods \cite{GreengardRokhlin1987}, treecodes \cite{BarnesHut1986}, and fast-Fourier-transform methods \cite{PhillipsWhite1997}.
In our numerical solution (developed as the open-source code \pygbe), we accelerate the $N$-body calculation with a treecode \cite{BarnesHut1986,LiJohnstonKrasny2009}, making this part of the algorithm scale as $O(N\log N)$ rather than $O(N^2)$. 

The treecode algorithm groups the sources and targets in a tree-structured set of boxes and approximates interactions between far-away boxes using a series expansion---a Taylor series, in our case. This allows for controllable accuracy that depends on the number of terms used in the expansion and the multipole-acceptance criterion that defines the threshold where the distance between source and target is far enough to approximate the interactions with expansions. The interactions of targets with close-by sources and expansion centers are computationally intensive calculations, which \pygbe offloads to the \gpu. Details of our implementation of the treecode in \pygbe can be found in our previous work \cite{CooperBarba-share154331}.
 %discretization and treecode

\subsection{Energy calculation} \label{sec:energy}
%!TEX root = CooperBarba2014.tex

We can decompose the total free energy into Coulombic, surface, and solvation energy:

\begin{equation}
F_\text{Total} = F_\text{Coulomb} + F_\text{surf} + F_\text{solv}.
\end{equation}

\medskip

\paragraph*{Coulombic energy---}

The Coulombic energy arises simply from the Coulomb interactions of all point charges. We compute it by

\begin{equation} \label{eq:coul_energy}
F_\text{Coulomb} = \frac{1}{2} \sum_j^{N_q}\sum^{N_q}_{\substack{i\\ i\neq j}} q_iq_j\frac{1}{4\pi |\mathbf{r}_i-\mathbf{r}_j|}
\end{equation}

\paragraph*{Solvation free energy---}

The solvation energy is the energy contribution of the protein's surroundings: solvent polarization, charged surfaces, and other proteins. We compute it as

\begin{align} \label{eq:solv_energy}
F_{\text{solv}} &= \frac{1}{2} \int_{\Omega} \rho \,(\phi_{\text{total}} - \phi_{\text{Coulomb}}) \\
&= \sum_{k=0}^{N_q} q_k (\phi_{\text{total}} - \phi_{\text{Coulomb}})(\mathbf{r}_k),
\end{align}

\noindent where $\rho$ is the charge distribution, consisting of point charges (which transforms the integral into a sum), and $\phi_\text{reac} = \phi_{\text{total}} - \phi_{\text{Coulomb}}$ is
%
\begin{equation} \label{eq:phi_reac_bem}
\phi_{\text{reac},\mathbf{r}_k} = -K_{L}^{\mathbf{r}_k}(\phi_{1,\Gamma_1}) + V_{L}^{\mathbf{r}_k} \left(\frac{\partial}{\partial \mathbf{n}}\phi_{1,\Gamma_1} \right) 
\end{equation}

\paragraph*{Surface free energy---}

We use the derivation of free energy of a surface with prescribed charge (like $\Gamma_2$ in Figure \ref{fig:molecule_surface}) from Chan and co-workers \cite{ChanMitchell1983,CarnieChan1993}. They describe the free energy on a surface as

\begin{equation} \label{eq:energy_surf}
F_\text{surf} = \frac{1}{2} \int_{\Gamma} G_c \sigma_0^2 d\Gamma, 
\end{equation} 

\noindent where $\phi = G_c \sigma_0$.


%=============


\section{Analytical solution} \label{sec:analytical_solution}
%!TEX root = CooperBarba2014.tex

It is possible to derive a closed-form expression for the free energy of interaction between a spherical molecule with a centered charge and a spherical surface with imposed potential or charge, like the one sketched in Figure \ref{fig:twosphere_an}.  There are such analytical expressions for interacting charged surfaces \cite{CarnieChanGunning1994}, and interacting spherical molecules with multiple point charges inside \cite{LotanHead-Gordon2006}, but not for a situation where surfaces and molecules interact. Having such an analytical solution is of great utility in the development of a computational model for protein-surface interaction, because it will allow for proper code verification. 


\subsection{Expansion in Legendre polynomials} \label{sec:expansion_analytical}


The system of partial differential equations from Equation \eqref{eq:pde}  models the electrostatic potential field in the setting of Figure \ref{fig:twosphere_an}. Following Carnie and co-workers \cite{CarnieChanGunning1994}, the axial symmetry lets us formulate the solution of Equation \eqref{eq:pde} as an expansion in Legendre polynomials:
 

\begin{align} \label{eq:derivation1}
\phi_1 = \sum_{n=0}^{\infty} c_n r_1^n P_n(\cos \theta_1) & + \frac{q}{4\pi\epsilon_1 r_1} \quad \text{on $\Omega_1$,} \nonumber \\
\phi_2 = \sum_{n=0}^{\infty} a_n k_n(\kappa r_1) P_n (& \cos \theta_1) \nonumber \\
+ \sum_{n=0}^{\infty} b_n k_n & (\kappa r_2) P_n(\cos \theta_2) \quad \text{ on $\Omega_2$,}
\end{align}

\noindent being $P_n$ the $n^{\text{th}}$-degree Legendre polynomial and $k_n$ the modified spherical Bessel function of the second kind. 

 

\begin{figure}%[h] 
   \centering
   \includegraphics[width=0.45\textwidth]{Figure7.pdf} 
   \caption{Sketch of system solved with Legendre polynomials expansions.}
   \label{fig:twosphere_an}
\end{figure}
 

We make use of the following addition formula \cite{MarceljaMitchellNinhamSculley1977},
%
\begin{equation} \label{eq:addition_formula}
k_n(\kappa r_2) P_n(\cos \theta_2) = \sum_{m=0}^{\infty}(2m+1) B_{nm} i_m(\kappa r_1) P_m(\cos \theta_1),
\end{equation}
%
\noindent to reformulate the expression for $\phi_2$ in Equation \eqref{eq:derivation1} as 
%
\begin{align} \label{eq:derivation2}
\phi_2 =& \sum_{n=0}^{\infty} a_n k_n(\kappa r_1) P_n(\cos \theta_1) \nonumber \\
& + \sum_{n=0}^{\infty} b_n \sum_{m=0}^{\infty}(2m+1) B_{nm} i_m(\kappa r_1) P_m(\cos \theta_1) \nonumber \\ 
\phi_2 =& \sum_{n=0}^{\infty} b_n k_n(\kappa r_2) P_n(\cos \theta_2) \nonumber \\
& + \sum_{n=0}^{\infty} a_n \sum_{m=0}^{\infty}(2m+1) B_{nm} i_m(\kappa r_2) P_m(\cos \theta_2).
\end{align}

Here, $i_m$ is the modified spherical Bessel function of the first kind; $B_{nm}$ is defined by 

\begin{equation} \label{eq:Bnm}
B_{nm} = \sum_{\nu=0}^{\infty} A_{nm}^{\nu} k_{n+m-2\nu}(\kappa R),
\end{equation}

\noindent where $R$ is the center-to-center distance; and $A_{nm}^{\nu}$ is given by the following expression, with ${\bf \Gamma}$ (in this context only) representing the gamma function:

%\begin{widetext}
\begin{equation} \label{eq:Anm}
A_{nm}^{\nu} = \frac{{\bf \Gamma}(n-\nu+0.5){\bf \Gamma}(m-\nu+0.5){\bf \Gamma}(\nu+0.5)(n+m-\nu)!(n+m-2\nu+0.5)}{\pi {\bf \Gamma}(m+n-\nu+1.5)(n-\nu)!(m-\nu)!\nu!}.
\end{equation}
%\end{widetext}



Legendre polynomials are orthogonal to each other, and $\frac{q}{4\pi\epsilon_1 r_1}$ is independent of $\theta$. Thus, taking the inner product of the expressions in Equations \eqref{eq:derivation1} and  \eqref{eq:derivation2} with $P_j(\cos \theta_i)$, where $i=1$ or $2$, yields
%
\begin{equation} \label{eq:derivation3}
\phi_1\delta_{0j} = c_j r_1^j + \frac{q}{4\pi\epsilon_1 r_1} \delta_{0j}  
\end{equation}
\noindent for the first expression of Equation \eqref{eq:derivation1}, and
%
\begin{align} \label{eq:derivation3.5}
\phi_2\delta_{0j} = &a_j k_j(\kappa r_1) + \sum_{n=0}^{\infty} b_n(2j+1)B_{nj} i_j(\kappa r_1),  \nonumber \\
\phi_2\delta_{0j} = &b_j k_j(\kappa r_2) + \sum_{n=0}^{\infty} a_n(2j+1)B_{nj} i_j(\kappa r_2)  
\end{align}
\noindent for Equation \eqref{eq:derivation2}.

Applying the interface conditions for $\Gamma_1$ on Equation \eqref{eq:derivation3} and the first expression of Equation \eqref{eq:derivation3.5}, produces
%
\begin{align}\label{eq:derivation4}
\sum_{n=0}^{\infty} a_n \left( \kappa k_{n}'(\kappa d_1) - \frac{\epsilon_1}{\epsilon_2} \frac{n}{d_1} k_n(\kappa d_1) \right) \delta_{nj} +& \nonumber \\ 
b_n (2j+1)B_{nj} \left( \kappa i_{j}'(\kappa d_1) - \frac{\epsilon_1}{\epsilon_2} \frac{j}{d_1} i_j(\kappa d_1)  \right) & = \nonumber \\
-\frac{\epsilon_1}{\epsilon_2} \frac{q}{4\pi\epsilon_1 d_1^2} \delta_{0j}&(j+1),
\end{align}

\noindent where $d_1$ is the radius of surface $1$.

\subsubsection*{Constant potential $\phi$ on $\Gamma_2$.}
The application of the boundary condition on $\Gamma_2$, $\phi(\Gamma_2) = \phi_0$, where $\phi_0$ is independent on $\theta_2$, gives

\begin{equation} \label{eq:derivation5_phi}
\sum_{n=0}^{\infty} a_n(2j+1)B_{nj}i_j(\kappa d_2) + b_nk_n(\kappa d_2) \delta_{nj} = \phi_0 \delta_{0j}.
\end{equation}

\noindent Combining Equations \eqref{eq:derivation4} and  \eqref{eq:derivation5_phi} yields the following system of equations for the coefficients $a_n$ and $b_n$
%
\begin{align} \label{eq:system_phi}
\mathbf{I} \mathbf{A} + \mathbf{L} \mathbf{B} &= -\frac{\epsilon_1}{\epsilon_2} \frac{q}{4\pi\epsilon_1 d_1^2} \mathbf{e} \nonumber \\
\mathbf{M} \mathbf{A} + \mathbf{I} \mathbf{B} &= \phi_0 \mathbf{e}
\end{align}

\noindent where
%
\begin{align} \label{eq:phi_terms}
I_{jn} &= \delta_{jn} \nonumber \\
e_j &= \delta_{0j} \nonumber \\
A_n &= a_n \left(\kappa k_n'(\kappa d_1) - \frac{\epsilon_1}{\epsilon_2} \frac{n}{d_1} k_n(\kappa d_1) \right) \nonumber \\
B_n &= b_n k_n(\kappa d_2) \nonumber \\
L_{jn} &= (2j+1)B_{nj}\left( \kappa \frac{i_j'(\kappa d_1)}{k_n(\kappa d_2)} - \frac{\epsilon_1}{\epsilon_2} \frac{j}{d_1} \frac{i_j(\kappa d_1)}{k_n(\kappa d_2)} \right) \nonumber \\
M_{jn} &= (2j+1)B_{nj} i_j(\kappa d_2) \frac{1}{\left(\kappa k_n'(\kappa d_1) - \frac{\epsilon_1}{\epsilon_2} \frac{n}{d_1} k_n(\kappa d_1) \right)}. 
\end{align}

\subsubsection*{Constant surface charge $\sigma$ on $\Gamma_2$.}
In this case, the application of the boundary condition on $\Gamma_2$, $\sigma(\Gamma_2) = -\epsilon_2 \frac{\partial \phi}{\partial \mathbf{n}} \Large|_{\Gamma_2} = \sigma_0$, where $\sigma_0$ is independent on $\theta_2$, gives

\begin{equation} \label{eq:derivation5_dphi}
\sum_{n=0}^{\infty} a_n(2j+1)B_{nj}\kappa i_j'(\kappa d_2) + b_n \kappa k_n'(\kappa d_2) \delta_{nj} = -\frac{\sigma_0}{\epsilon_2} \delta_{0j}
\end{equation}

\noindent Combining Equations \eqref{eq:derivation4} and  \eqref{eq:derivation5_phi} produces a system of equations for the coefficients $a_n$ and $b_n$
%
\begin{align} \label{eq:system_dphi}
\mathbf{I} \mathbf{A} + \mathbf{L} \mathbf{B} &= -\frac{\epsilon_1}{\epsilon_2} \frac{q}{4\pi\epsilon_1 d_1^2} \mathbf{e} \nonumber \\
\mathbf{M} \mathbf{A} + \mathbf{I} \mathbf{B} &= -\frac{\sigma_0}{\epsilon_2} \mathbf{e}
\end{align}

\noindent where
%
\begin{align} \label{eq:dphi_terms}
I_{jn} &= \delta_{jn} \nonumber \\
e_j &= \delta_{0j} \nonumber \\
A_n &= a_n \left(\kappa k_n'(\kappa d_1) - \frac{\epsilon_1}{\epsilon_2} \frac{n}{d_1} k_n(\kappa d_1) \right) \nonumber \\
B_n &= b_n \kappa k_n'(\kappa d_2) \nonumber \\
L_{jn} &= (2j+1)B_{nj}\left( \frac{i_j'(\kappa d_1)}{k_n'(\kappa d_2)} - \frac{\epsilon_1}{\epsilon_2} \frac{j}{d_1} \frac{i_j(\kappa d_1)}{\kappa k_n'(\kappa d_2)} \right) \nonumber \\
M_{jn} &= (2j+1)B_{nj} \kappa i_j'(\kappa d_2) \frac{1}{\left(\kappa k_n'(\kappa d_1) - \frac{\epsilon_1}{\epsilon_2} \frac{n}{d_1} k_n(\kappa d_1) \right)}. 
\end{align}
 
\subsection{Energy calculation} \label{energy_analytical}


 \medskip
 \paragraph*{Solvation free energy of the molecule---}
According to Equation \eqref{eq:solv_energy}, the solvation free energy of a molecule with a centered charge is given by
%
\begin{equation} \label{eq:energy_phi}
F_{\text{solv}} = \frac{1}{2} q \phi_{\text{reac}}(r_1=0),
\end{equation} 
 
 \noindent and using Equation \eqref{eq:derivation1}, the reaction potential from Equation \eqref{eq:phi_reac_bem} is:
 %
 \begin{equation} \label{eq:phi_reac_an}
 \phi_{\text{reac}} = \phi - \frac{q}{4\pi\epsilon_1 r} = \sum_{n=0}^{\infty} c_n r^n P_n(\cos \theta_1).
 \end{equation}
 
 Applying the boundary conditions at $\Gamma_1$ on Equation  \eqref{eq:derivation3}, we can rewrite $c_j$ in terms of the already computed $a_j$ and $b_j$:
 %
 \begin{align}
 c_j = \frac{1}{d_1^j} & \Big(a_j k_j(\kappa d_1) + \nonumber \\
&  \sum_{m=0}^{\infty} b_m(2j+1)B_{mj} i_j(\kappa d_1) - \frac{q}{4\pi\epsilon_1 d_1} \delta_{0j} \Big)
 \end{align} 
 
Because the charge is located at $r=0$, only the $n=0$ terms of Equation \eqref{eq:phi_reac_an} will survive, and the potential at this location is:
 
 \begin{align} \label{eq:phi_reac_an2}
 \phi_{\text{reac}} (r_1=0) = & a_0 k_0(\kappa d_1) + \nonumber \\
 &\sum_{m=0}^{\infty} b_m B_{m0}i_0(\kappa d_1) - \frac{q}{4\pi\epsilon_1 d_1}
 \end{align}
 
 The result from Equation \eqref{eq:phi_reac_an2} in Equation \eqref{eq:energy_phi} yields the solvation free energy. 
 
For the isolated molecule, $R \to \infty$ makes $B_{nm} \to 0$, which nullifies the sum in Equation \eqref{eq:phi_reac_an2} and $a_0$ for $R \to \infty$, from the system in Equation \eqref{eq:system_phi}, is 

\begin{equation} \label{eq:a0_inf}
a_0^{\infty} = -\frac{q}{d_1^2}\frac{\epsilon_1}{\epsilon_2} \frac{1}{4\pi\kappa k_0'(\kappa d_1) \epsilon_1}
\end{equation}


\medskip
\paragraph*{Surface  free energy with set potential $\phi_0$---}
We can expand $G_p$ from Equation \eqref{eq:energy_surf} in Legendre polynomials as
%
\begin{align} \label{eq:G_p}
G_p = &-\frac{\epsilon_2 \kappa}{\phi_0}  \Bigg[ \sum_{n=0}^{\infty} b_n k_n'(\kappa d_2) P_n(\cos \theta_2) \nonumber \\ 
& + \sum_{n=0}^{\infty} a_n \sum_{m=0}^{\infty} (2m+1) B_{nm} i_m'(\kappa d_2) P_m(\cos \theta_2) \Bigg].
\end{align}

 \noindent Applying Equation \eqref{eq:G_p} in Equation \eqref{eq:energy_surf} gives

\begin{equation} \label{G_p_int}
F = 2\pi \kappa \phi_0 d_2^2 \epsilon_2 \left[ b_0 k_0'(\kappa d_2) + \sum_{n=0}^{\infty} a_n B_{n0} i_0'(\kappa d_2) \right]
\end{equation}

 \noindent If the surface is isolated, $R \to \infty$ makes $B_{n0} \to 0$, and the free energy in this case is 
%
\begin{equation} \label{energy_isolated_phi}
F = 2\pi \kappa \phi_0 d_2^2 b_0^{\infty} k_0'(\kappa d_2) \epsilon_2
\end{equation}
 
 \noindent where $b_0^{\infty}$ is taken from the system in  \eqref{eq:system_phi} considering $B_{nm} \to 0$, which results in
 %
 \begin{equation} \label{b_inf_phi}
 b_0^{\infty} = \frac{\phi_0}{k_0(\kappa d_2)}.
 \end{equation}
 
 \medskip
 \paragraph*{Surface  free energy with set charge $\sigma_0$---}
We can expand $G_c$ from Equation \eqref{eq:energy_surf} in Legendre polynomials as
%
\begin{align} \label{eq:G_c}
G_c = & \frac{1}{\sigma_0} \Bigg[ \sum_{n=0}^{\infty} b_n k_n(\kappa d_2) P_n(\cos \theta_2) + \nonumber \\ 
&\sum_{n=0}^{\infty} a_n \sum_{m=0}^{\infty} (2m+1) B_{nm} i_m(\kappa d_2) P_m(\cos \theta_2) \Bigg]
\end{align}

 \noindent Applying Equation \eqref{eq:G_c} into Equation \eqref{eq:energy_surf} gives

\begin{equation} \label{G_c_int}
F = 2\pi \sigma_0 d_2^2 \left[ b_0 k_0(\kappa d_2) + \sum_{n=0}^{\infty} a_n B_{n0} i_0(\kappa d_2) \right]
\end{equation}

 \noindent For the isolated surface, $R \to \infty$ and $B_{n0} \to 0$, and the free energy is 
%
\begin{equation} \label{energy_isolated_dphi}
F = 2\pi \sigma_0 d_2^2 b_0^{\infty} k_0(\kappa d_2) 
\end{equation}
 
 \noindent where $b_0^{\infty}$ is calculated from the system in  \eqref{eq:system_dphi} considering $B_{nm} \to 0$, which results in
 
 \begin{equation} \label{b_inf_dphi}
 b_0^{\infty} = -\frac{\sigma_0}{\epsilon_2 \kappa k_0'(\kappa d_2)}.
 \end{equation}


\section{Results} \label{sec:results}
%!TEX root = CooperBarba2014.tex

To obtain the following results, we extended the \pygbe code to consider surfaces with prescribed charge or potential. For all runs, we used a workstation with Intel Xeon X5650 \cpu s  and one \nvidia Tesla C2075 \gpu\ card (2011 Fermi). We used the free \msms software \cite{SannerOlsonSpehner1995} to generate meshes, and \texttt{pdb2pqr} \cite{Dolinsky04} with an \amber forcefield to determine the charges and van der Waals radii. 

\subsection{Verification against analytical solution} \label{sec:verification}

Using the analytical solution detailed in Section \ref{sec:analytical_solution}, we carried out a grid-convergence study of \pygbe extended to treat interacting surfaces with biomolecules. The setup consists of a spherical molecule with a $5$\AA~radius and a centered charge of $1e^-$, interacting with a spherical surface of $4$\AA~radius and an imposed potential of $\phi=1$. The center-to-center distance between the spheres is $12$\AA, and they are dissolved in water with salt at 145mM, which gives a Debye length of 8 ($\kappa = 0.125$), and permittivity $\epsilon_\text{sol} = 80$. The permittivity inside the spherical protein is $\epsilon_\text{mol} = 4$. Figure \ref{fig:twosphere_num} shows a sketch of this system.

\begin{figure}[h] %  figure placement: here, top, bottom, or page
   \centering
   \includegraphics[width=0.45\textwidth]{Figure8.pdf} 
   \caption{Sketch of system used in the convergence study of Figure \ref{fig:error_sphere}.}
   \label{fig:twosphere_num}
\end{figure}

Figure \ref{fig:error_sphere} presents the results of the grid-convergence analysis, where the error is the relative difference in interaction free energy between the analytical result from Section \ref{sec:analytical_solution} and the numerical solution computed with \pygbe. The observed order of convergence of the three finest meshes was 1.007. Table \ref{table:params1} presents the numerical parameters used in this case. Recall from section \ref{sec:methods} that we calculate the boundary-element integrals differently for close-by and far-away elements, and use a semi-analytical method for the element that contains the collocation point. The fine Gauss quadrature rule is used for elements closer than $2L$ from the collocation point, where $L=\sqrt{2\cdot \text{Area}}$. For the treecode,  $N_{\text{crit}}$ is the maximum number of boundary elements per box, $P$ is the Taylor expansion truncation parameter and $\theta$ is the multipole-acceptance criterion. The final numerical parameter is the exit tolerance of the \textsc{gmres} solver.

\begin{table}[h]
  %\centering
   %\fontfamily{ppl}\selectfont
   \caption{\label{table:params1}Numerical parameters used in the code-verification runs with the analytical solution. } 
    \begin{tabular}{c c c c c c c}
	\hline%\toprule
	\multicolumn{3}{l} {\# Gauss points:} & \multicolumn{3}{l}{Treecode:} & \gmres:\\
	\footnotesize{in-element} & \footnotesize{close-by} & \footnotesize{far-away} & $N_{\text{crit}}$ & $P$ &  $\theta$  & tol.\\
	\hline%\midrule
	9 per side & 37 & 3  &  300 & 15 & 0.5  & $10^{-9}$\\	
	\hline%\bottomrule
    \end{tabular}
\end{table}


\begin{figure}[htbp] %  figure placement: here, top, bottom, or page
   \centering
   \includegraphics[width=0.4\textwidth]{Figure9.pdf} 
   \caption{Grid-convergence study for the interaction free energy between a spherical molecule with a centered charge and a sphere with potential $\phi=1$. Data sets, figure files plus running/plotting scripts are available under \ccby \cite{CooperBarba2015-share1348841}}.
   \label{fig:error_sphere}
\end{figure}

As seen in Figure \ref{fig:error_sphere}, the error decays with the average area of the boundary elements ($\frac{1}{N}$), which is the expected behavior considering our previous work \cite{CooperBarba-share154331}. This proves that the extension of \pygbe to treat charged surfaces is solving the mathematical model correctly.

%The interaction free energy involves three separate calculations: one with both bodies (molecule and interacting surface with set potential) and one for each isolated body. The time to solution for each mesh in Figure \ref{fig:time} corresponds to the total time, including all three cases that need to be computed. The most time consuming part of the algorithm is the matrix-vector product within the Krylov solver, which scales as $O(N \log N)$ thanks to the treecode acceleration. However, the total time to solution scales slightly worse than $O(N \log N)$. This happens because the condition number of the system depends on the number of elements and more iterations are required to converge to the specified tolerance, which is revealed by Figure \ref{fig:iterations}.

\subsection{Protein G B1 D4$^{\prime}$} \label{sec:PGB}

We computed the electrostatic field of protein G B1 D4$^{\prime}$ interacting with a 100\AA$\times$100\AA$\times$10\AA\ block with surface charge density $0.05$C/m$^2$. 
The protein was centered with respect to a  100\AA$\times$100\AA\ face, a distance 2\AA\ above it. Since we did not consider any Stern layers or solvent-filled cavities, these tests contain only two surfaces: the protein's \ses and the charged surface.
We also computed the electrostatic field generated by protein G B1 D4$^\prime$ and the surface by themselves.

The angle between the dipole moment of the protein and the vector normal to the surface was $\alpha_\text{tilt}=10^\circ$. 
The dipole-moment vector placed at the center of mass of the protein generates an axis, and we used the line of shortest distance between the outermost atom and this axis as a reference vector $\mathbf{V}_{\text{ref}}$. 
The rotation angle $\alpha_{\text{rot}}$ is the angle between the normal vector to a 100\AA$\times$10\AA\ side face of the block and $\mathbf{V}_{\text{ref}}$ when $\alpha_\text{tilt}=0$, and is equal to $200^\circ$ in these tests. Figure \ref{fig:protein_surface} is a sketch of this arrangement.

\begin{figure}[h] %  figure placement: here, top, bottom, or page
   \centering
   \includegraphics[width=0.35\textwidth]{Figure4.pdf} 
   \caption{Orientation of a protein near a charged surface. $\mathbf{m}$ is the dipole moment vector, $\mathbf{V}_\text{ref}$ the vector between $\mathbf{m}$ and the atom that is the furthest, and $\mathbf{n}_1$ and $\mathbf{n}_2$ are normal to their corresponding surfaces. $\alpha_\text{tilt}$ is the angle between $\mathbf{n}_1$ and $\mathbf{m}$, and $\alpha_\text{rot}$ the angle between $\mathbf{V}_\text{ref}$ and $\mathbf{n}_2$ when $\mathbf{m}$ and $\mathbf{n}_1$ are aligned.}
   \label{fig:protein_surface}
\end{figure}

In this case, the solvent has no salt, i.e., $\kappa=0$, and its relative permittivity was 80. The region inside the protein had a relative permittivity of 4.

We computed the solvation and surface energy using meshes with 1, 2, 4 and 8 elements per square Angstrom with the parameters detailed in Table \ref{table:params2}. 
Using Richardson extrapolation and the result of the three finest meshes we calculated an approximate exact solution, shown in Table \ref{table:extraPGB}, which we consider as a reference to calculate estimated errors. 
These errors are the relative difference between the energy obtained with Richardson extrapolation and the results with each mesh, and we plot them in Figures \ref{fig:convergence_1PGB_isolated} and \ref{fig:convergence_1PGB_sensor}. 
Those figures show errors that are decaying as $1/N$ in both the solvation and surface energies for the finest three meshes. This indicates that the calculations are in the asymptotic region and the geometry is well resolved in these cases.
In fact, the observed order of convergence was 0.95 for the solvation energy and 1.12 for the surface energy in the isolated case, and 0.96 for the solvation energy and 0.94 for the surface energy when the protein and surface were interacting. 
Using the extrapolated values from Table \ref{table:extraPGB}, we obtain an interaction free energy of $-7.6$ [kcal/mol].
For details on the Richardson-extrapolation method for performing grid-convergence analysis, see our previous work \cite{CooperBardhanBarba2013}.

\begin{table}[h]
  %\centering
   %\fontfamily{ppl}\selectfont
   \caption{\label{table:params2}Numerical parameters used in the convergence runs with protein G B1 D4$^{\prime}$. } 
    \begin{tabular}{c c c c c c c}
	\hline%\toprule
	\multicolumn{3}{l} {\# Gauss points:} & \multicolumn{3}{l}{Treecode:} & \gmres:\\
	\footnotesize{in-element} & \footnotesize{close-by} & \footnotesize{far-away} & $N_{\text{crit}}$ & $P$ &  $\theta$  & tol.\\
	\hline%\midrule
	9 per side & 19 & 7  &  500 & 15 & 0.5  & $10^{-8}$\\	
	\hline%\bottomrule
    \end{tabular}
\end{table}


\begin{table}[h]
  %\centering
   %\fontfamily{ppl}\selectfont
   \caption{\label{table:extraPGB}Extrapolated values of energy for protein G B1 D4$^\prime$.} 
    \begin{tabular}{c c c}
	\hline%\toprule
	& \multicolumn{2}{c} {Energy [kcal/mol]} \\
	& Solvation & Surface \\
	\hline%\midrule
    Isolated    & $-218.26$ & $317.41$ \\
	Interacting & $-222.43$ & $313.98$ \\	
	\hline%\bottomrule
    \end{tabular}
\end{table}

\begin{figure}[h] %  figure placement: here, top, bottom, or page
   \centering
   \includegraphics[width=0.45\textwidth]{convergence_isolated.pdf} 
   \caption{Grid-convergence study of the solvation energy for an isolated protein G B1 D4$^{\prime}$ mutant, and the surface energy of an isolated surface with charge density of 0.05C/m$^2$.}
   \label{fig:convergence_1PGB_isolated}
\end{figure}

\begin{figure}[h] %  figure placement: here, top, bottom, or page
   \centering
   \includegraphics[width=0.45\textwidth]{Figure10.pdf} 
   \caption{Grid-convergence study of the solvation and surface energy for protein G B1 D4$^{\prime}$ mutant, interacting with a surface with a charge density of 0.05C/m$^2$. Data sets, figure files and running/plotting scripts available under \ccby \cite{CooperBarba2015-share1348803}}.
   \label{fig:convergence_1PGB_sensor}
\end{figure}


\subsection{Reproducibility and data management}
To facilitate the reproduction of our work, we consistently release the code and data associated with every publication. In that context, \pygbe was released under an MIT open-source license with our previous publication \cite{CooperBardhanBarba2013}, through a version-control repository. 
In this case, we prepared \emph{``reproducibility packages''} containing running and post-processing scripts in Python to generate Figures \ref{fig:error_sphere} and \ref{fig:convergence_1PGB_sensor}. The packages invoke \pygbe with the parameters and meshes reported here, and then produce the plots, all with a single command.
The reproducibility packages are hosted on \textbf{figshare}, and are referenced in the respective captions.

\section{Discussion} \label{sec:discussion}
%!TEX root = CooperBarba2014.tex

\subsection{First case: protein G B1 D4$^\prime$} \label{sec:disc_1PGB}

The orientation of protein G B1 D4$^\prime$ near charged surfaces was studied using molecular dynamics (MD) simulations by Liu and co-workers\cite{LiuLiaoZhou2013} and  experimentally by Baio and co-workers.\cite{BaioWeidnerBaughGambleStaytonCastner2012} The availability of these published results was a motivation to test \pygbe using this protein. 

The results presented in Figure \ref{fig:1PGB_probability} show that for the most likely orientations, the dipole-moment vector is aligned with the vector normal to the interacting surface. This indicates that the dipole moment is the dominant effect that determines the protein's orientation, over local protein-surface  interactions. This result is unsurprising, since protein G B1 D4$^\prime$ is a relatively small biomolecule. 

Moreover, Figure \ref{fig:1PGB_probability} reveals that protein G B1 D4$^\prime$ behaves like a point dipole, as the most likely orientations shift 180$^\circ$ when the sign of the surface charge is flipped. This is also explained by the dipole moment dominating the orientation.

The dipolar behavior described by our calculations with \pygbe agrees with the experiments done by Baio and co-workers, \cite{BaioWeidnerBaughGambleStaytonCastner2012} in which they observed opposite orientations of protein G B1 D4$^\prime$ adsorbed on NH$_3^+$ and COO$^-$ self-assembled monolayers. With positively charged surfaces, most of the proteins oriented with the N-terminal of the protein pointing away from the surface, while for negatively charged surfaces the opposite occurred, with the C-terminal pointing away from the surface. This agrees with our results in Figure \ref{fig:1PGB_probability} since the dipole moment vector of protein G B1 D4$^\prime$ points from the C-terminal to the N-terminal.

Liu and co-workers \cite{LiuLiaoZhou2013} used MD simulations to obtain $<\cos(\alpha_{\text{tilt}})>=0.95$ for $\sigma = 0.05$C/m$^2$, and $<\cos(\alpha_{\text{tilt}})>=-0.85\pm0.05$ for $\sigma = -0.05$C/m$^2$, which agrees well with our results in Table \ref{table:avg}. MD simulations consider van der Waals interactions and conformational changes of the protein, whereas these are not considered in our approach, explaining the slight differences in $<\cos(\alpha_{\text{tilt}})>$.

 \subsection{Second case: immunoglobulin G}
 
 With the extension of \pygbe verified with an analytical solution\cite{CooperBarba2015a} and confirmation that the implicit-solvent model can be used to study protein-surface interaction with a small protein (Section \ref{sec:disc_1PGB}), we proceeded to explore the effect of surface charge and salt concentration on the orientation of the antibody immunoglobulin G. Antibodies are widely used in biosensors as ligand molecules, due to their affinity and specificity with the target molecule (antigen), and it is vitally important that they are adsorbed on the sensor with the fragment antigen-binding (Fab) pointing away from the sensor, into the incoming flow containing the antigens.
 
 Figures \ref{fig:1IGT_negcharge} and \ref{fig:1IGT_poscharge} present the probability distribution of immunoglobulin G for many orientations (given by $\alpha_\text{tilt}$ and $\alpha_\text{rot}$) varying the surface charge ($\sigma$) and salt concentration ($\kappa$). Figures \ref{fig:1IGT_2D_sig-005} and \ref{fig:1IGT_2D_sig005} show that for low surface charge ($\sigma \pm 0.05$C/m$^2$) and high salt concentration ($\kappa=0.125$\AA$^{-1}$), there is no clear preferred orientation, to the point that the most likely orientation has a probability of around 10\%. This means that adsorbing the antibodies under these conditions would result in a wide range of orientations, which is not favorable for biosensor fabrication.
 
 \medskip
 
 \paragraph*{Effect of surface charge---}
 
 With greater surface charge, in this case $\sigma=\pm0.2$C/m$^2$, the orientation probability distribution gets narrower for positive surface charge, and is maintained for negative surface charge. Figure \ref{fig:1IGT_2D_sig020_kappa01250} shows a much clearer preferred orientation, with a probability more than 5$\times$ higher for positive surface charge at high salt concentration. For low salt concentrations (Figures \ref{fig:1IGT_2D_sig-020_kappa003125} and \ref{fig:1IGT_2D_sig020_kappa003125}), this effect is even larger. 
 
The results presented on Figures \ref{fig:1IGT_negcharge} and \ref{fig:1IGT_poscharge} also show that increasing the surface charge has very little effect on the dipole moment orientation. This is evidence that, in contrast to the case of protein G B1 D4$^\prime$, local interactions dominate over the dipole moment. If the dipole moment were the dominant effect, the dipole moment vector would tend to align to the surface normal as the surface charge increases.
 
 \medskip
 
 \paragraph*{Effect of salt concentration---}
 
 We also varied the Debye length ($\kappa^{-1}$) four-fold. In terms of salt concentration, it means a 16$\times$ decrease in the amount of salt. 
 
 Like increasing the surface charge, lowering the salt concentration narrows the orientation probability distribution. For $\sigma=\pm0.05$C/m$^2$ (Fig.~\ref{fig:1IGT_2D_sig-005_kappa003125} and Fig.~\ref{fig:1IGT_2D_sig005_kappa003125}), the effect on positive or negative surface charge is very similar: the preferred orientation is about 2$\times$ more likely. However, for $\sigma=\pm0.2$C/m$^2$, the increase is larger with negative surface charge (Fig.~\ref{fig:1IGT_2D_sig-020_kappa003125}), than with positive surface charge (Fig.~\ref{fig:1IGT_2D_sig020_kappa003125}). The narrower probability distribution is explained by the lower shielding effect caused by the reduced salt content, which at the same time increases the electrostatic interaction.

From the results in Figure \ref{fig:1IGT_negcharge} and Figure \ref{fig:1IGT_poscharge}, we can conclude that it is easier to control the antibody orientation with low salt concentration and high surface charge, because the orientation probability distribution is the narrowest. In our results, Figures \ref{fig:1IGT_3D_sig-02_kap003125_til124-rot140} and \ref{fig:1IGT_3D_sig02_kap003125_til076-rot160} show the orientation of the antibody at the lowest salt concentration and higher surface charge, but only the orientation in Figure \ref{fig:1IGT_3D_sig02_kap003125_til076-rot160}, with positive surface charge, is favorable for biosensing applications, since the Fab fragments are pointing up.

That favorable orientations for biosensing applications are best obtained with high positive surface charge and low salt concentration is consistent with experimental observations by Chen and co-workers. \cite{ChenLiuZhouJiang2003} These researchers developed a coarse-grained method known as the united residue model,\cite{ZhouChenJiang2003} which qualitatively aligns with our results.
 


\section{Conclusion}
%!TEX root = CooperBarba2014.tex

In this work, we used an implicit-solvent model to study protein-surface interaction. We present for the first time and apply an extension of our open-source \pygbe code to account for the presence of surfaces with imposed potential or charge. The new feature of the code was verified against an analytical solution, which we derived for that purpose. 

To demonstrate the power of this approach in a more realistic setting, we performed tests of protein G B1 D4$^\prime$ near a brick-shaped surface with an imposed charge. The error in energy scaling with the area of boundary elements demonstrates that this extension of \pygbe is capable of resolving the mathematical model correctly. This test was motivated by the biosensing application, where a ligand molecule is adsorbed on a \sam-coated nanoparticle, which can be represented by the brick-shaped surface.

The addition of a surface with imposed charge or potential in the implicit-solvent model falls naturally in a boundary integral approach. In this case, the region enclosed by the surface is not part of the domain, then, this surface only adds one equation to the linear system, rather than two, which is the case with the molecular solvent-excluded surface.

We conclude that this implicit-solvent model can offer a valuable approach in protein-surface interaction studies. This tool can be useful for orientation studies of ligand molecules in biosensors, either to find optimal adsorption conditions of salt concentration and surface charge, or to guide the design of better ligand molecules. 





\section*{Acknowledgments}
 This work was supported by ONR via grant \#N00014-11-1-0356 of the Applied Computational Analysis Program. LAB also acknowledges support from NSF CAREER award OCI-1149784 and from NVIDIA, Inc.\ via the CUDA Fellows Program. 
 We are grateful for many helpful conversations with members of the Materials and Sensors Branch of the Naval Research Laboratory, especially Dr. Jeff M. Byers and Dr. Marc Raphael.

% Create the reference section using BibTeX:
\bibliographystyle{elsarticle-num}
\bibliography{CompBio,bem,scicomp,fastmethods,scbib,biosensors}

\end{document}
