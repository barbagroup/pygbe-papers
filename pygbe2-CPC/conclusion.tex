%!TEX root = CooperBarba2014.tex

In this work, we used an implicit-solvent model to study protein-surface interaction. We present for the first time and apply an extension of our open-source \pygbe code to account for the presence of surfaces with imposed potential or charge. The new feature of the code was verified against an analytical solution, which we derived for that purpose. 

To demonstrate the power of this approach in a more realistic setting, we performed tests of protein G B1 D4$^\prime$ near a brick-shaped surface with an imposed charge. The error in energy scaling with the area of boundary elements demonstrates that this extension of \pygbe is capable of resolving the mathematical model correctly. This test was motivated by the biosensing application, where a ligand molecule is adsorbed on a \sam-coated nanoparticle, which can be represented by the brick-shaped surface.

The addition of a surface with imposed charge or potential in the implicit-solvent model falls naturally in a boundary integral approach. In this case, the region enclosed by the surface is not part of the domain, then, this surface only adds one equation to the linear system, rather than two, which is the case with the molecular solvent-excluded surface.

We conclude that this implicit-solvent model can offer a valuable approach in protein-surface interaction studies. This tool can be useful for orientation studies of ligand molecules in biosensors, varying salt concentration and surface charge to find the optimal adsorption conditions, or to guide the design of better ligand molecules. 
