%!TEX root = CooperBarba2014.tex

Interactions between proteins and solid surfaces are ubiquitous in many biological processes. Adsorption serves a key function in natural activities, like blood coagulation, and  in biotechnologies like tissue engineering, biomedical implants and biosensors.
A full understanding of protein-surface interactions has remained elusive \cite{Gray2004,RabeVerdesSeegel2011}, but adsorption mechanisms are governed by surface energy and often the dominant effect is electrostatics. As a free-energy-driven process, protein-surface interaction is difficult to study experimentally \cite{MijajlovicETal2013}, and thus simulations offer a good alternative. Full atomistic molecular dynamics simulations demand large amounts of computing effort, however, so we often must resort to other methods.

Protein electrostatics can be studied via modeling approaches using the Poisson-Boltzmann equation and implicit-solvent representations. These models  are popular for computing solvation energies in protein systems \cite{RouxSimonson1999,Bardhan2012}, but few studies have included the effect of surfaces. Lenhoff and co-workers studied surface-protein interactions using continuum models discretized with boundary-element \cite{YoonLenhoff1992,RothLenhoff1993,AsthagiriLenhoff1997} and finite-difference methods \cite{YaoLenhoff2004,YaoLenhoff2005}, in the context of ion-exchange chromatography. They realized that van der Waals effects can be neglected for realistic molecular geometries \cite{RothNealLenhoff1996} and that the model is adequate as long as conformational changes in the protein are slight \cite{YaoLenhoff2004,YaoLenhoff2005}. 

The aim of this work is to develop and assess a computational model to simulate proteins near engineered surfaces of fixed charge, using implicit-solvent electrostatics.
We have added the capability of modeling a protein near a charged surface to our code \pygbe, an open-source code\footnote{\url{https://github.com/barbagroup/pygbe}}  that solves the Poisson-Boltzmann equations via an integral formulation, using a fast multipole algorithm and \gpu\ hardware acceleration.  Previously, we verified and validated \pygbe in its use to obtain solvation and binding energies, by comparing with analytical solutions of the equations and with results obtained using the well-known \apbs software \cite{CooperBarba-share154331,CooperBardhanBarba2013}. 
In the present work, we derived an analytical solution for a spherical molecule interacting with a spherical surface of prescribed charge, and used it to verify the code in its new application and study numerical convergence.
Using the newly extended code, we also studied the interaction between protein \gb and a solid surface of imposed charge, 
and conduct a grid-convergence study using this more realistic surface geometry.

We intend our new modeling tool to be useful in studying the behavior of proteins as they adsorb on surfaces that have been functionalized with  self-assembled monolayers (\sam), which are modeled within an implicit-solvent framework as surfaces with prescribed charge. 
One application is biosensing, where the target molecules are captured on the sensor via ligand molecules (for which antibodies are a common choice). Favorable orientations of ligand molecules lead to greatly enhanced sensitivity of biosensors \cite{TajimaTakaiIshihara2011,TrillingBeekwilderZuilhof2013}, because binding sites need to be physically accessible to the targets. Studies of protein orientation near charged surfaces might look at how orientation can be influenced by engineering decisions regarding surface preparation, to aid the design of better biosensors. We explore this application in a companion publication that obtains probability of orientations for an antibody near a surface, as function of changing conditions on charge and ionic strength \cite{CooperBarba2015b}. 
In this paper, we present the details of a new analytical solution for spherical charged surfaces and molecules, grid-convergence studies for the interaction free energy in this case, and grid-convergence studies for protein \gb alone and interacting with a charged surface. The detailed analysis of the model is complemented with a diligent effort for reproducibility and we deposit both input and results data in accessible and permanent archival storage, in addition to the open-source code.
