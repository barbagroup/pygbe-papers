%!TEX root = CooperBarba2014.tex

Proteins interacting with surfaces and adsorption mechanisms are ubiquitous and play a role in many biological processes. 
Along its importance in natural activity, like blood coagulation, adsorption affects biotechnologies like tissue engineering, biomedical implants and biosensors.
Yet, despite their importance, a full understanding of protein-surface interactions remains elusive \cite{Gray2004,RabeVerdesSeegel2011}.

Implicit-solvent models using the Poisson-Boltzmann equation are popular for computing solvation energies in protein systems \cite{RouxSimonson1999,Bardhan2012}, but few studies have included the effect of surfaces. Lenhoff and co-workers studied surface-protein interactions using continuum models discretized with boundary-element \cite{YoonLenhoff1992,RothLenhoff1993,AsthagiriLenhoff1997} and finite-difference methods \cite{YaoLenhoff2004,YaoLenhoff2005}, in the context of ion-exchange chromatography. They realized that van der Waals effects can be neglected for realistic molecular geometries \cite{RothNealLenhoff1996} and that the model is adequate as long as conformational changes in the protein are slight \cite{YaoLenhoff2004,YaoLenhoff2005}. 

We have added the capability of modeling a protein near a charged surface to our code \pygbe, an open-source code\footnote{\url{https://github.com/barbagroup/pygbe}}  that uses \gpu\ hardware.  Previously, we verified and validated \pygbe in its use to obtain solvation and binding energies, by comparing with analytical solutions of the equations and with results obtained using the well-known \apbs software \cite{CooperBarba-share154331,CooperBardhanBarba2013}. 
In the present work, we derived an analytical solution for a spherical molecule interacting with a spherical charged surface, and used it to verify the code in its new application and study numerical convergence.
Using the newly extended code, we studied the interaction between protein GB1 D4' and a charged surface.

We anticipate this modeling tool to be useful for understanding the behavior of proteins as they adsorb on surfaces with self-assembled monolayers (\sam), which can be seen as a surface with imposed charge within an implicit-solvent model. 
One such application is biosensing, where the target molecule is attached to the sensor through a ligand molecule (in many cases an antibody). The performance of a biosensor is largely affected by the orientation of the ligand molecule \cite{TajimaTakaiIshihara2011,TrillingBeekwilderZuilhof2013}, as the binding sites must be accessible for the target molecule, and orientation studies using this tool can potentially aid the design of better ligand molecules. We explore this application in a companion publication \cite{CooperBarba2015b}.
