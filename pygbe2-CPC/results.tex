%!TEX root = CooperBarba2014.tex

To obtain the following results, we extended the \pygbe code to consider surfaces with prescribed charge or potential. For all runs, we used a workstation with Intel Xeon X5650 \cpu s  and one \nvidia Tesla C2075 \gpu\ card (2011 Fermi). We used the free \msms software \cite{SannerOlsonSpehner1995} to generate meshes, and \texttt{pdb2pqr} \cite{Dolinsky04} with an \amber forcefield to determine the charges and van der Waals radii. 

\subsection{Verification against analytical solution} \label{sec:verification}

Using the analytical solution detailed in Section \ref{sec:analytical_solution}, we carried out a grid-convergence study of \pygbe extended to treat interacting surfaces with biomolecules. The setup consists of a spherical molecule with a $5$\AA~radius and a centered charge of $1e^-$, interacting with a spherical surface of $4$\AA~radius and an imposed potential of $\phi=1$. The center-to-center distance between the spheres is $12$\AA, and they are dissolved in water with salt at 145mM, which gives a Debye length of 8 ($\kappa = 0.125$), and permittivity $\epsilon_\text{sol} = 80$. The permittivity inside the spherical protein is $\epsilon_\text{mol} = 4$. Figure \ref{fig:twosphere_num} shows a sketch of this system.

\begin{figure}[h] %  figure placement: here, top, bottom, or page
   \centering
   \includegraphics[width=0.45\textwidth]{Figure8.pdf} 
   \caption{Sketch of system used in the convergence study of Figure \ref{fig:error_sphere}.}
   \label{fig:twosphere_num}
\end{figure}

Figure \ref{fig:error_sphere} presents the results of the grid-convergence analysis, where the error is the relative difference in interaction free energy between the analytical result from Section \ref{sec:analytical_solution} and the numerical solution computed with \pygbe. The observed order of convergence of the three finest meshes was 1.007. Table \ref{table:params1} presents the numerical parameters used in this case. Recall from section \ref{sec:methods} that we calculate the boundary-element integrals differently for close-by and far-away elements, and use a semi-analytical method for the element that contains the collocation point. The fine Gauss quadrature rule is used for elements closer than $2L$ from the collocation point, where $L=\sqrt{2\cdot \text{Area}}$. For the treecode,  $N_{\text{crit}}$ is the maximum number of boundary elements per box, $P$ is the Taylor expansion truncation parameter and $\theta$ is the multipole-acceptance criterion. The final numerical parameter is the exit tolerance of the \textsc{gmres} solver.

\begin{table}[h]
  %\centering
   %\fontfamily{ppl}\selectfont
   \caption{\label{table:params1}Numerical parameters used in the code-verification runs with the analytical solution. } 
    \begin{tabular}{c c c c c c c}
	\hline%\toprule
	\multicolumn{3}{l} {\# Gauss points:} & \multicolumn{3}{l}{Treecode:} & \gmres:\\
	\footnotesize{in-element} & \footnotesize{close-by} & \footnotesize{far-away} & $N_{\text{crit}}$ & $P$ &  $\theta$  & tol.\\
	\hline%\midrule
	9 per side & 37 & 3  &  300 & 15 & 0.5  & $10^{-9}$\\	
	\hline%\bottomrule
    \end{tabular}
\end{table}


\begin{figure}[htbp] %  figure placement: here, top, bottom, or page
   \centering
   \includegraphics[width=0.4\textwidth]{Figure9.pdf} 
   \caption{Grid-convergence study for the interaction free energy between a spherical molecule with a centered charge and a sphere with potential $\phi=1$. Data sets, figure files plus running/plotting scripts are available under \ccby \cite{CooperBarba2015-share1348841}}.
   \label{fig:error_sphere}
\end{figure}

As seen in Figure \ref{fig:error_sphere}, the error decays with the average area of the boundary elements ($\frac{1}{N}$), which is the expected behavior considering our previous work \cite{CooperBarba-share154331}. This proves that the extension of \pygbe to treat charged surfaces is solving the mathematical model correctly.

%The interaction free energy involves three separate calculations: one with both bodies (molecule and interacting surface with set potential) and one for each isolated body. The time to solution for each mesh in Figure \ref{fig:time} corresponds to the total time, including all three cases that need to be computed. The most time consuming part of the algorithm is the matrix-vector product within the Krylov solver, which scales as $O(N \log N)$ thanks to the treecode acceleration. However, the total time to solution scales slightly worse than $O(N \log N)$. This happens because the condition number of the system depends on the number of elements and more iterations are required to converge to the specified tolerance, which is revealed by Figure \ref{fig:iterations}.

\subsection{Protein G B1 D4$^{\prime}$} \label{sec:PGB}

We computed the electrostatic field of protein G B1 D4$^{\prime}$ interacting with a 100\AA$\times$100\AA$\times$10\AA\ block with surface charge density $0.05$C/m$^2$. 
The protein was centered with respect to a  100\AA$\times$100\AA\ face, a distance 2\AA\ above it. Since we did not consider any Stern layers or solvent-filled cavities, these tests contain only two surfaces: the protein's \ses and the charged surface.
We also computed the electrostatic field generated by protein G B1 D4$^\prime$ and the surface by themselves.

The angle between the dipole moment of the protein and the vector normal to the surface was $\alpha_\text{tilt}=10^\circ$. 
The dipole-moment vector placed at the center of mass of the protein generates an axis, and we used the line of shortest distance between the outermost atom and this axis as a reference vector $\mathbf{V}_{\text{ref}}$. 
The rotation angle $\alpha_{\text{rot}}$ is the angle between the normal vector to a 100\AA$\times$10\AA\ side face of the block and $\mathbf{V}_{\text{ref}}$ when $\alpha_\text{tilt}=0$, and is equal to $200^\circ$ in these tests. Figure \ref{fig:protein_surface} is a sketch of this arrangement.

\begin{figure}[h] %  figure placement: here, top, bottom, or page
   \centering
   \includegraphics[width=0.35\textwidth]{Figure4.pdf} 
   \caption{Orientation of a protein near a charged surface. $\mathbf{m}$ is the dipole moment vector, $\mathbf{V}_\text{ref}$ the vector between $\mathbf{m}$ and the atom that is the furthest, and $\mathbf{n}_1$ and $\mathbf{n}_2$ are normal to their corresponding surfaces. $\alpha_\text{tilt}$ is the angle between $\mathbf{n}_1$ and $\mathbf{m}$, and $\alpha_\text{rot}$ the angle between $\mathbf{V}_\text{ref}$ and $\mathbf{n}_2$ when $\mathbf{m}$ and $\mathbf{n}_1$ are aligned.}
   \label{fig:protein_surface}
\end{figure}

In this case, the solvent has no salt, i.e., $\kappa=0$, and its relative permittivity was 80. The region inside the protein had a relative permittivity of 4.

We computed the solvation and surface energy using meshes with 1, 2, 4 and 8 elements per square Angstrom with the parameters detailed in Table \ref{table:params2}. 
Using Richardson extrapolation and the result of the three finest meshes we calculated an approximate exact solution, shown in Table \ref{table:extraPGB}, which we consider as a reference to calculate estimated errors. 
These errors are the relative difference between the energy obtained with Richardson extrapolation and the results with each mesh, and we plot them in Figures \ref{fig:convergence_1PGB_isolated} and \ref{fig:convergence_1PGB_sensor}. 
Those figures show errors that are decaying as $1/N$ in both the solvation and surface energies for the finest three meshes. This indicates that the calculations are in the asymptotic region and the geometry is well resolved in these cases.
In fact, the observed order of convergence was 0.95 for the solvation energy and 1.12 for the surface energy in the isolated case, and 0.96 for the solvation energy and 0.94 for the surface energy when the protein and surface were interacting. 
Using the extrapolated values from Table \ref{table:extraPGB}, we obtain an interaction free energy of $-7.6$ [kcal/mol].
For details on the Richardson-extrapolation method for performing grid-convergence analysis, see our previous work \cite{CooperBardhanBarba2013}.

\begin{table}[h]
  %\centering
   %\fontfamily{ppl}\selectfont
   \caption{\label{table:params2}Numerical parameters used in the convergence runs with protein G B1 D4$^{\prime}$. } 
    \begin{tabular}{c c c c c c c}
	\hline%\toprule
	\multicolumn{3}{l} {\# Gauss points:} & \multicolumn{3}{l}{Treecode:} & \gmres:\\
	\footnotesize{in-element} & \footnotesize{close-by} & \footnotesize{far-away} & $N_{\text{crit}}$ & $P$ &  $\theta$  & tol.\\
	\hline%\midrule
	9 per side & 19 & 7  &  500 & 15 & 0.5  & $10^{-8}$\\	
	\hline%\bottomrule
    \end{tabular}
\end{table}


\begin{table}[h]
  %\centering
   %\fontfamily{ppl}\selectfont
   \caption{\label{table:extraPGB}Extrapolated values of energy for protein G B1 D4$^\prime$.} 
    \begin{tabular}{c c c}
	\hline%\toprule
	& \multicolumn{2}{c} {Energy [kcal/mol]} \\
	& Solvation & Surface \\
	\hline%\midrule
    Isolated    & $-218.26$ & $317.41$ \\
	Interacting & $-222.43$ & $313.98$ \\	
	\hline%\bottomrule
    \end{tabular}
\end{table}

\begin{figure}[h] %  figure placement: here, top, bottom, or page
   \centering
   \includegraphics[width=0.45\textwidth]{convergence_isolated.pdf} 
   \caption{Grid-convergence study of the solvation energy for an isolated protein G B1 D4$^{\prime}$ mutant, and the surface energy of an isolated surface with charge density of 0.05C/m$^2$.}
   \label{fig:convergence_1PGB_isolated}
\end{figure}

\begin{figure}[h] %  figure placement: here, top, bottom, or page
   \centering
   \includegraphics[width=0.45\textwidth]{Figure10.pdf} 
   \caption{Grid-convergence study of the solvation and surface energy for protein G B1 D4$^{\prime}$ mutant, interacting with a surface with a charge density of 0.05C/m$^2$. Data sets, figure files and running/plotting scripts available under \ccby \cite{CooperBarba2015-share1348803}}.
   \label{fig:convergence_1PGB_sensor}
\end{figure}
