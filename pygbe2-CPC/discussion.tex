%!TEX root = CooperBarba2014.tex

%\subsection{Verification with analytical solution} \label{sec:disc_analytical}
In order to study the interaction of proteins and charged surfaces, we extended \pygbe to account for surfaces with prescribed charge or potential. Unfortunately, there was no analytical solution available in the literature to compare and verify \pygbe's extension. 
Section \ref{sec:analytical_solution} derives a closed-form expression for a spherical molecule with a centered charge interacting with a spherical nanosurface with imposed charge or potential.
We used this new analytical solution to conduct a grid-convergence study of the interaction energy (Figure \ref{fig:error_sphere}). The error decays with the area, which is the expected behavior \cite{CooperBardhanBarba2013, CooperBarba-share154331} for a boundary element method with constant elements. 
Discretization error is very small for a spherical geometry. To make sure that the errors due to integration, the treecode approximation and the \textsc{gmres} solver were even smaller, we chose all the numerical parameters for high accuracy. This allows us to observe the convergence with respect to the discretization only and extract the order. With more realistic molecular geometries, however, discretization errors will be larger and the requirements with \pygbe can be relaxed in the other numerical parameters, resulting in lower runtimes.

%\subsection{Protein G B1 D4$^\prime$ near a charged surface} \label{sec:disc_1PGB}

The results in Figures \ref{fig:convergence_1PGB_isolated} and \ref{fig:convergence_1PGB_sensor} show the applicability of this approach in more realistic situations. The setting in Figure \ref{fig:protein_surface} can model a nanostructure coated with a self-assembled monolayer (\sam) interacting with a protein that will adsorb on that surface. Our application of interest in developing this model is the field of biosensors, where it can assist design through studies of electrostatic adsorption affecting protein orientation near the biosensor. With antibody-based biosensors, orientation determines the accessibility of reaction sites and is critical for sensitivity. In a companion publication \cite{CooperBarba2015b}, we present the first studies of protein orientation near charged surfaces using our modeling framework.

Figures \ref{fig:convergence_1PGB_isolated} and \ref{fig:convergence_1PGB_sensor} show the expected 1/N convergence with a simple protein. Just like in the case of the sphere, the simple geometry of the charged surface forced us to use very fine parameters in order to see the correct order of convergence. If we needed to perform this computation many times, for example, to sample different protein orientations, we might relax these parameters to obtain shorter computation times. 

The extrapolated values in Table \ref{table:extraPGB} are useful to find more relaxed parameters that still give acceptable results. For example, using a mesh density of 2 elements per square Angstrom on the charged surface and 4 elements per square Angstrom on the protein, and the parameters detailed in Table \ref{table:params3}, we get the results in Table \ref{table:relaxPGB}. These results are less than 2\% away from those in Table \ref{table:extraPGB}, and each run takes less than one minute. Moreover, using the energy values from Table \ref{table:relaxPGB}, the interaction free energy is $-7.61$ [kcal/mol], which is very close to the extrapolated case ($-7.6$ [kcal/mol]).

\begin{table}[h]
  %\centering
   %\fontfamily{ppl}\selectfont
   \caption{\label{table:params3}Numerical parameters for relaxed runs with protein \gb. } 
    \begin{tabular}{c c c c c c c}
    \hline%\toprule
    \multicolumn{3}{l} {\# Gauss points:} & \multicolumn{3}{l}{Treecode:} & \gmres:\\
    \footnotesize{in-element} & \footnotesize{close-by} & \footnotesize{far-away} & $N_{\text{crit}}$ & $P$ &  $\theta$  & tol.\\
    \hline%\midrule
    9 per side & 19 & 1  &  300 & 4 & 0.5  & $10^{-5}$\\
    \hline%\bottomrule
    \end{tabular}
\end{table}

\begin{table}[h]
  %\centering
   %\fontfamily{ppl}\selectfont
   \caption{\label{table:relaxPGB}Values of energy for protein \gb using the parameters in Table \ref{table:params3}, and a mesh density of 4 elements per square angstrom in the protein and 2 elements per square angstrom on the charged surface}
    \begin{tabular}{c c c}
    \hline%\toprule
    & \multicolumn{2}{c} {Energy [kcal/mol]} \\
    & Solvation & Surface \\
    \hline%\midrule
    Isolated    & $-221.56$ & $315.33$ \\
    Interacting & $-225.81$ & $311.96$ \\
    \hline%\bottomrule
    \end{tabular}
\end{table}
